O desenvolvimento do aplicativo móvel e da plataforma Web \appName\ foi feito utilizando diversas tecnologias e ferramentas, assim como metodologias. A seguir neste capítulo, serão descritas cada uma das etapas do processo de criação do \textit{app}.

\section{Processo}

%%% trocar para referências de processo de desenvolvimento
O processo de \textit{software} \cite{processos_software} é visto por uma sequência de atividades que produzem uma variedade de documentos, resultando em um programa satisfatório e executável. Cada vez mais, os processos se mostram tão importantes quanto o \textit{software} em si, sendo indispensáveis para a produção de um produto de maior qualidade. O desenvolvimento da aplicação foi feito utilizando a metodologia ágil Scrum \cite{scrum_guide}.

O movimento ágil da indústria de \textit{software} \cite{agile_software_development} se iniciou com a publicação do manifesto ágil\footnote{\url{https://agilemanifesto.org/}} em 2001 por um grupo de praticantes e consultores de \textit{software} . O manifesto agrega os seguintes valores:

\begin{itemize}
    \item \textbf{Indivíduos e Interações} acima de processos e ferramentas;
    \item \textbf{\textit{Software} em funcionamento} acima de uma documentação compreensiva;
    \item \textbf{Colaboração com o cliente} acima de negociações de contrato;
    \item \textbf{Responder a mudanças} mais que seguir um plano;
\end{itemize}

Scrum \cite{scrum_guide} é um \textit{framework} que ajuda as pessoas, times e organizações a gerar valor através de soluções adaptáveis para problemas complexos. De forma resumida, o Scrum requer uma pessoa denominada \textit{Scrum Master} para controlar um ambiente caracterizado por:

\begin{enumerate}
    \item O dono do produto (\textit{Product Owner}): solicita o trabalho para um item complexo que é enviado para o \textit{Product Backlog};
    \item O time: transforma os itens selecionados em subitens de valor incremental durante uma fase denominada \textit{Sprint};
    \item O time, juntamente com as partes interessadas: avaliam os resultados e fazem os planejamentos e ajustes necessários para a próxima \textit{Sprint};
    \item Processo se repete com a nova \textit{Sprint}.
\end{enumerate}

O Scrum aplicado no \appName\ foi adaptado às necessidades do projeto. Foram usadas \textit{Sprints} para as diversas fases do projeto, incluindo a revisão das tarefas concluídas e o planejamento das próximas tarefas, onde as tarefas do \textit{backlog} são selecionadas para entrar no escopo da \textit{Sprint} conforme necessidade. É comum no Scrum, a ocorrência de reuniões diárias para que a situação das tarefas sejam compartilhadas entre a equipe e os impedimentos sejam expostos para que seja buscada uma solução, porém, por se tratar de uma equipe de apenas duas pessoas e sem a presença de um \textit{Scrum Master}, a frequência das reuniões foi diminuida, ocorrendo, no mínimo uma vez por semana.

O \textit{framework} Scrum não tem uma lista de artefatos de documentação obrigatórios a se produzir, pois se trata de uma metodologia ágil, caracterizada pelo contato pessoal e o funcionamento do produto acima de uma documentação expressiva. Porém, as documentações não são proibidas, e devem ser produzidas conforme a necessidade da equipe e do cliente, servindo como critérios de aceite ou como referência interna para auxiliar no desenvolvimento ou manutenções futuras. Para o projeto, foram desenvolvidos artefatos conforme a necessidade surgia.

Diversos recursos e programas foram utilizados para auxiliar nos processos de desenvolvimento do \appName\, sendo todos de licença livre ou de uso gratuito.
%%% TODO: informar a licença
Antes de iniciar o desenvolvimento do \textit{software}, foram criados protótipos de telas do projeto, a fim de facilitar a criação das telas funcionais posteriormente. Durante esta etapa, o estilo do aplicativo foi definido, incluindo as cores predominantes, diferentes telas pertencentes à interface de usuário, rotas de navegação e as funções acessíveis por cada uma das telas projetadas. Tudo isso conferiu uma identidade ao projeto de \textit{software}. A seguir, estão descritos os demais artefatos produzidos durante o desenvolvimento do \appName:

\begin{itemize}
    \item Protótipos de telas (\ref{figma});
    \item Fluxo de utilização da aplicação (\ref{lucidChart});
    \item Histórias de Usuário (\ref{hus});
    \item Diagrama de classes (\ref{plantuml}).
\end{itemize}

Com o protótipo inicial criado e aprovado, foi utilizado um \textit{software} de organização de tarefas chamado \textit{Notion} (\ref{notion}) para realizar o cadastro dos itens a serem implementados no aplicativo e foram criadas histórias de usuário para documentar os critérios de aceite e as funcionalidades a serem desenvolvidas. Com o início da primeira \textit{Sprint}, os primeiros \textit{cards} pendentes de resolução foram divididos entre a dupla e foi iniciado o desenvolvimento da plataforma em questão.

A seguir, algumas das ferramentas utilizadas no processo de planejamento e desenvolvimento do \appName\ são descritas, assim como os artefatos produzidos em cada uma delas.

\subsection{Figma}
\label{figma}

Segundo Connel e Shaffer \cite{prototyping}, um protótipo de \textit{software} é um modelo dinâmico visual que oferece uma ferramenta de comunicação para o cliente e para o desenvolvedor que é bem mais efetiva do que qualquer prosa narrativa ou modelos estáticos visuais para representar funcionalidade. O \textit{software} Figma\footnote{\url{https://www.figma.com/}} é uma plataforma de design hospedada em formato de aplicação \textit{Web} que facilita a criação de protótipos para aplicativos \textit{mobile}, \textit{web}, e muitos outros.

A criação dos protótipos foi feita utilizado o Figma de forma gratuita. A plataforma permite a criação de conta para o uso gratuito, porém também possui planos pagos de assinatura. Com as ferramentas disponibilizadas pelo figma, foram prototipadas todas as telas do aplicativo de acordo com a ideia inicial do projeto, assim como o fluxo de navegação por parte do usuário final. O protótipo pode ser dividido em 4 partes:

\subsubsection{Autenticação}

Composta por 3 telas: Login, Registro e \textit{Reset} de Senha. As telas são navegáveis entre si por meio dos \textit{links} clicáveis de cor azul ou cinza.

\figura[H]{../img/figma_auth.png}{Protótipos de telas pertencentes às rotas de autenticação}{figma_auth}{width=0.75\textwidth}

\subsubsection{Cursos}

Definitivamente a parte mais importante do aplicativo, sendo composta por \refFig{figma_course1} e \refFig{figma_course2}. Foram prototipadas 6 telas para este item, incluindo a tela inicial chamada de \textit{Pré-Home}, onde o usuário é redirecionado após efetuar o \textit{login}. A tela \textit{Home} é carregada por meio do botão para acessar a área de cada curso, onde se encontram os principais dados do curso, como os tópicos e o \textit{ranking}. Ao acessar um tópico a partir da tela \textit{Home}, o usuário é redirecionado para a tela deste tópico, onde pode selecionar um subtópico para resolver desafios ou visualizar o \textit{ranking} local.

\figura[H]{../img/figma_course1.png}{Protótipos de telas contendo a \textit{Home} do aplicativo e seleção de Tópicos}{figma_course1}{width=0.75\textwidth}

Pela \refFig{figma_course2}, é possível visualizar a tela do subtópico com as opções de acessar os desafios disponíveis e o ranking. A tela de desafio contém um enunciado com uma imagem opcional e campos para resposta. Por fim, a tela de classificação vai conter os usuários ordenados pelo melhor aproveitamento do curso.

\figura[H]{../img/figma_course2.png}{Protótipos de telas contendo a página de tópico com opções para nevegação para os desafios cadastrados e para o ranking}{figma_course2}{width=0.75\textwidth}

\subsubsection{\textit{Chat}}

Acessível pela barra de navegação, a tela de \textit{Chats} contém todas as pessoas com as quais o usuário logado se comunicou pelo aplicativo. Conforme mostrado na \refFig{figma_chat}, a tela de conversa é carregada ao selecionar um dos usuários listados.

\figura[H]{../img/figma_chat.png}{Protótipos de telas contendo as funções de envio e recebimanto de mensagens no \appName}{figma_chat}{width=0.5\textwidth}

\subsubsection{Gerenciamento de Conta}

Rota composta por 3 telas mostradas na \refFig{figma_account}, acessíveis pela barra de navegação, sendo uma para visualização dos dados de usuário, com opções para alteração dos dados e para fazer o \textit{Logout}, enquanto as outras são para realizar a alteração efetiva dos dados pessoais e senha.

\figura[H]{../img/figma_account.png}{Protótipos de telas pertencentes às rotas de gerenciamento de conta}{figma_account}{width=0.75\textwidth}

\subsection{Markdown}
\label{hus}

Markdown \cite{markdown} é uma linguagem de marcação de texto leve para a criação de textos formatados utilizando um simples editor de texto. As histórias de usuário da aplicação foram criadas utilizando a linguagem Markdown conforme o exemplo a seguir:

\textbf{Como} usuário, 
\textbf{eu quero} ser capaz de fazer login com as minhas credenciais no sistema, 
\textbf{para} acessar as funcionalidades da plataforma.

Critério de Aceitação 01:
\begin{enumerate}
    \item Usuário digita as suas credenciais corretamente e pressiona no botão de login;
    \item Usuário acessa a ferramenta normalmente com a sua conta.
\end{enumerate}

Criério de Aceitação 02:
\begin{enumerate}
    \item Usuário digita as credenciais incorretas e pressiona o botão de login;
    \item Acesso negado e mensagem de erro explicativa é mostrada na tela.
\end{enumerate}

As demais histórias de usuário podem ser acessadas por meio do repositório auxiliar\footnote{\url{https://pedenite.github.io/monografia-pages/user_stories.pdf}} criado para o projeto no GitHub (\ref{git}).

Markdown também foi usado no aplicativo e aplicação \textit{Web} para a criação e exibição dos desafios, de forma a permitir ao professor uma maior versatilidade. Para deixar a aplicação mais acessível a professores que não têm conhecimentos sobre a linguagem, foi usado um editor mais visual no \textit{Back Office}, com as opções de formatação em botões e uma visualização do texto formatado ao lado.

\subsection{\textit{Lucid Chart}}
\label{lucidChart}

O \textit{Lucid Chart}\footnote{\url{https://www.lucidchart.com/}} é uma ferramenta \textit{online} com plano gratuito para realizar o desenho de fluxos, podendo ser feito de forma colaborativa. Com as ferramentas disponibilizadas pela plataforma, foi criado o fluxograma da \refFig{fluxo_app} para o aplicativo móvel, que mostra as funcionalidades acessíveis pelo usuário final por cada tela. Também é mostrado o funcionamento básico dos desafios incluídos na plataforma, que fornecem uma pontuação ao usuário somente quando a pergunta é respondida corretamente, não incrementando os pontos quando desafio é respondido incorretamente.

\figura[H]{../img/fluxo_app.png}{Fluxo de utilização do aplicativo móvel \appName\ na visão do usuário}{fluxo_app}{width=\textwidth}

Também foi desenvolvido um fluxograma para a aplicação \textit{Web} conforme mostrado na \refFig{fluxo_web}

\figura[H]{../img/fluxo_web.png}{Fluxo de utilização da aplicação \textit{Web} \appName\ na visão do usuário}{fluxo_web}{width=\textwidth}

Pela \refFig{fluxo_app} e \refFig{fluxo_web}, é possível visualizar o início e o fim do fluxo de usabilidade do \textit{Front Office} e \textit{Back Office} em cor vermelha, sendo o início, o \textit{login} e o fim, o \textit{logout}. As telas foram represantadas por retângulos de cor azul e as condições por losangos de cor azul claro. Por fim, funcionalidades específicas são representadas pelo item de cor amarelada.

\subsection{PlantUML}
\label{plantuml}

\textit{Unified Modeling Language} (UML) \cite{uml} é uma linguagem de modelagem usada no campo de engenharia de \textit{software} que visa criar uma maneira padrão de visualizar o \textit{design} de um sistema. Dentre os tipos de diagramas que fazem parte do UML, estão o diagrama de classe, de objeto, de sequência, de caso de uso, de atividade, entre muitos outros. Para o contexto do projeto, foi gerado o diagrama de classes seguindo o UML.

PlantUML \footnote{\url{https://plantuml.com/}} é uma ferramenta que permite escrever diagramas UML por meio de uma sintaxe textual. Ele também suporta outros diversos diagramas que não fazem parte do UML. A ferramenta foi usada para gerar o diagrama de classes da aplicação conforme \refFig{class_diagram}.

\figura[H]{../img/class_diagram.png}{Diagrama de classes do \appName}{class_diagram}{width=\textwidth}

Pelo diagrama de classes, é possível visualizar as classes utilzadas no projeto, tanto na versão Front-office quanto Back-office, que refletem diretamente na modelagem do banco de dados não-relacional utilizado. Devido à categoria de banco de dados utilizada (\ref{firebase}), não foram gerados documentos adicionais para uma descrição mais visual da base.

\subsection{Notion}
\label{notion}

O software \textit{Notion}\footnote{\url{https://www.notion.so/}} é uma plataforma de produtividade utilizada principalmente para organizar trabalhos em equipe. A ferramenta dispõe de diversos recursos relacionados a metodologias ágeis, documentação, métricas, entre outros, porém para o \appName, foi utilizado o quadro kanban para a divisão e organização das tarefas do projeto, assim como para acompanhar as etapas do processo do Scrum.

\subsection{Git e Git Flow}
\label{git}

Git\footnote{\url{https://git-scm.com/}} \cite{git} é uma ferrameta ce controle de versão particularmente poderosa e flexível que torna o devenvolvimento colaborativo mais eficiente. A ferramenta foi inventada por Linus Torvalds para suportar o desenvolvimento do \textit{kernel} do Linux, mas se mostrou de grande valor para uma vasta gama de projetos.

Durante todo o desenvolvimento do projeto, foi utilizada a ferramenta Git para o controle de versão do código fonte. Para o compartilhamento do código e diferentes versões entre a dupla, foi usado o \textit{GitHub}\footnote{\url{https://github.com/}} como repositório remoto. A plataforma é de uso livre e funciona no formato \textit{Web}, facilitando o uso do Git, assim como acrescentando funcionalidades de organização de projetos que ajudam muito nos processos de desenvolvimento. Foram criados quatro repositórios no total para o projeto no GitHub de forma a promover uma melhor separação de cada plataforma e permitir até permitir o reuso de recursos com o repositório auxiliar.

\begin{enumerate}
    \item Front Office\footnote{\url{https://github.com/KesleyK/monografia-app}};
    \item Back Office\footnote{\url{https://github.com/KesleyK/monografia-backoffice/tree/master}};
    \item Monografia\footnote{\url{https://github.com/Pedenite/Monografia}};
    \item Auxiliar\footnote{\url{https://github.com/Pedenite/monografia-pages}}.
\end{enumerate}

O primeiro repositório listado é o que armazena todo o código-fonte do aplicativo móvel, assim como o segundo, que armazena o código da aplicação \textit{Web}. O terceiro item, intitulado Monografia contém todos os arquivos para gerar esta monografia e o último foi usado para incluir artefatos adicionais do projeto como páginas \textit{Web} que incluem a política de privacidade, PDFs com as histórias de usuário, entre outros. O repositório auxiliar utiliza de uma funcionalidade do GitHub chamada \textit{GitHub Pages} \cite{github-pages}, que permite a publicação de páginas \textit{Web} diretamente do repositório. Isso permitiu que as políticas de privacidade do aplicativo fossem publicados como um \textit{site}\footnote{\url{https://pedenite.github.io/monografia-pages/}}, assim como os outros artefatos incluídos.

Adicionalmente, para aprimorar o uso efetivo do Git, foi utilizado um fluxo de criação de \textit{branches} intitulado \textit{Git Flow}, que consiste de regras de nomenclatura e de hierarquia para as ramificações criadas em um projeto. Na \refFig{gitflow} é possível visualizar as nomenclaturas das \textit{branches} utilizadas no projeto \appName, assim como de onde são feitos os \textit{checkouts} e \textit{merges} para cada uma.

\figura[H]{../img/gitflow.png}{Exemplo visual do Git Flow}{gitflow}{width=\textwidth}

\begin{itemize}
    \item \textit{Master}: é a principal e não deve ter commits feitos diretamente nela, sendo usada para as \textit{releases} do projeto. 
    \item \textit{Hotfix}: é usada quando são descobertos erros na \textit{Master}, para corrigir o mais rápido possível.
    \item \textit{Develop}: vai acompanhar a \textit{Master}, inclusive sendo gerada a partir dela no início do projeto, e a partir dela que são geradas as \textit{branches} de \textit{Feature} e \textit{Bugfix}, caso necessário.
    \item \textit{Bugfix}: usadas para corrigir problemas encontrados na \textit{branch Develop} antes de realizar o \textit{merge} com a \textit{Master}, que pode representar um ambiente de produção, assim evitando problemas bem mais catastróficos.
    \item \textit{Feature}: representam os recursos incrementais do projeto como novas funcionalidades ou melhorias. Foi o tipo de \textit{branch} mais usado no projeto, sendo criada uma \textit{branch} para cada \textit{card} que não represente um defeito no Notion.
\end{itemize}

O uso do gitflow permitiu uma organização melhor do projeto para a realização do desenvolvimento de diversas funcionalidades simultaneamente sem interferência de uma nas outras. Com o uso dos \textit{Pull requests} do Github, foi possível aprimorar ainda mais o processo de \textit{merge} com a \textit{branch Develop} após o desenvolvimento de funcionalidades incluindo o \textit{Code Review} para a aprovação do item, permitindo uma melhor integração da dupla e conhecimento sobre o \textit{status} do projeto.

\section{Tecnologias Utilizadas}

Para o desenvolvimento do projeto de \textit{software}, diversas tecnologias foram utilizadas, incluindo linguagens de programação, \textit{frameworks}, ferramentas e ambientes. Nas seções seguintes, estão descritas cada uma delas.

\subsection{Android}

Android\footnote{\url{https://www.android.com/}} é o sistema operacional para \textit{smartphones} mais utilizado na atualidade, possuindo uma quota de mercado de 71.96\% segundo dados de Novembro de 2022 \cite{mobile-os}. Trata-se de um sistema \textit{open-source} baseado em Linux composto por uma pilha criada para suportar uma vasta quantidade de dispositivos e aspectos dos mesmos \cite{android}. A hierarquia do sistema android pode ser vista da seguinte forma:

\begin{itemize}
    \item \textbf{Aplicativos do sistema}: Todos os aplicativos que interagem diretamente com o usuário como calculadora, câmera, e-mail, entre outros;
    \item \textbf{Java API Framework}: Provedores de conteúdo, sistema de views e gerenciadores responsáveis por manter os aplicativos funcionando para o usuário final;
    \item \textbf{Bibliotecas C/C++ nativas e Runtime do Android}: Bibliotecas e funcionalidades requeridas pelos componentes e serviços do Android como o OpenGL e \textit{Android Core Libraries};
    \item \textbf{Hardware Abstract Layer (HAL)}: Camada de abstração de \textit{hardware}, é responsável por garantir o correto funcionamento das funcionalidades disponibilizadas pelo Android por meio de qualquer equipamento equipado ao aparelho como câmera, microfone, saída de audio, \textit{bluetooth}, entre outros.
    \item \textbf{Linux Kernel}: A fundação de todo o sistema Android, fornece funcionalidades de segurança, concorrência, gerenciamento de memória, entre muitos outros recursos de baixo nível para todas as camadas acima. Por ser um produto que já está no mercado a algumas décadas, também facilita para os fabricantes desenvolverem aparelhos para um \textit{Kernel} mais conhecido.
\end{itemize}

Todo o código do aplicativo \appName\ foi testado no sistema Android, porém, devido a utilização do React Native (\ref{react-section}) no código-fonte, o aplicativo é capaz de executar em dispositivos iOS também. Devido a falta de testes para a plataforma do iOS, não há garantias do correto funcionamento de todas as funcionalidades, ao contrário da versão para sistemas Android.

\subsection{Typescript}

Typescript\footnote{\url{https://www.typescriptlang.org/}} é uma linguagem de programação \textit{open-source} fortemente tipada que foi desenvolvida e é mantida pela Microsoft e é construída em cima da linguagem Javascript, permitindo um melhor uso ferramental a qualquer escala \cite{typescript}. Por sua vez, Javascript \cite{javascript} é uma linguagem leve, interpretada ou compilada \textit{just-in-time} com funções de primeira classe. É uma linguagem multi-paradigma que é usada vastamente em páginas da \textit{Web}, mas também tem ganhado uma grande popularidade em sistemas \textit{back-end} e \textit{Mobile} com tecnologias como Node.js e React Native.

A escolha da linguagem se deu por razões como a facilidade do uso e devido às opções que o typescript proporciona em relação à tipagem, que têm a capacidade de aumentar a produtividade ao se utilizar editores de texto avançados, assim como diminuir a quantidade de erros em tempo de execução.

\subsection{React e React Native}
\label{react-section}

React\footnote{\url{https://reactjs.org/}} é uma biblioteca Javascript para realizar a criação de interfaces de usuário. Trata-se de um projeto \textit{open-source} criado e mantido pela empresa Meta. A ideia geral da biblioteca é ser declarativa, baseada em componentes e multi-plataforma, permitindo a criação de um código modularizado mais simples de debugar, e que pode ser convertido para outras plataformas de forma relativamente fácil \cite{react}.

A sintaxe do React utiliza a extensão de sintaxe do Javascript chamada JSX, que se assemelha muito com a sintaxe de linguagens de marcação de texto como HTML e XML, porém traduzindo a sintaxe para códigos Javascript. Por este motivo, a biblioteca pode ser aprendida com uma certa facilidade por desenvolvedores que já estejam familiarizados com o desenvolvimento web nativo usando HTML.

React Native\footnote{\url{https://reactnative.dev}} compartilha dos mesmos princípios, funcionalidades e sintaxe do React para o desenvolvimento de aplicativos nativos. Os componentes presentes no código escrito em react native são transformados em elementos de interfaces de usuário utilizando a API da própria plataforma onde será executado. O \textit{framework} facilita o desenvolvimento para plataformas móveis de forma a utilizar o mesmo código tanto no Android como no iOS, não exigindo muito retrabalho ao criar suporte a outras plataformas para o aplicativo. Assim como React, o React Native é uma biblioteca \textit{open-source} mantida pela Meta \cite{react-native}.

A escolha do React e do React Native para o projeto \appName\ foi devido à familiaridade com elas por parte da dupla, não exigindo um treinamento extensivo para seu uso no projeto. A aplicação móvel, chamada de \textit{Front Office} foi desenvolvida usando React Native, enquanto a versão \textit{Web}, \textit{Back Office}, foi desenvolvida utilizando React.

\subsection{Expo}

Expo\footnote{\url{https://expo.dev/}} é uma plataforma \textit{open-source} para a criação de aplicativos nativos para Android, iOS e Web com Javascript e React \cite{expo}. A plataforma foi usada na versão \textit{mobile} do \appName\ para facilitar o uso de tecnologias nativas do Android, mas também facilitaria para o iOS, porém este não foi testado e, portanto não existe garantias de funcionamento.

Com o uso do Expo, foi também instalado um aplicativo chamado \textit{Expo Go}\footnote{\url{https://play.google.com/store/apps/details?id=host.exp.exponent}} para facilitar o teste durante o desenvolvimento. O aplicativo em questão permite a execução de aplicativos que utilizam a plataforma expo sem a necessidade de instalá-los no dispositivo, bastando acessar um link disponibilidado pela ferramenta ou ler um QR Code com a câmera do dispositivo, permitindo também a distribuição do \textit{app} por este meio. O aplicativo do \appName\ foi disponibilizado para execução por meio do aplicativo Expo Go por um link gerado pela ferramenta\footnote{\url{https://expo.dev/@pedenite/algoritmia?serviceType=classic&distribution=expo-go}}, permitindo a distribuição para testes.

\subsection{Jest}

\textit{Softwares} devem ser testados para transmitir a confiabilidade, ou seja, garantir o correto funcionamento nos ambientes intencionados \cite{test_automation}. A automação dos testes pode reduzir significativamente o esforço necessário para a realização adequada dos testes, assim como aumentar a quantidade de testes realizados durante um espaço de tempo limitado.

Jest\footnote{\url{https://jestjs.io/pt-BR/}} é um \textit{framework} de testes em javascript que tem como foco, a simplicidade \cite{jest}. É uma ferramenta que tem uma boa documentação e apresenta suporte para projetos que utilizam typescript e react, dentre vários outros. O Jest é mantido pela Meta sob uma licença \textit{open-source}, logo, o seu uso é liberado gratuitamente, assim como o seu código-fonte.

A escolha do Jest como ferramenta de testes se deu principalmente pelo suporte nativo do typescript e do React, que são tecnologias que foram utilizadas no desenvolvimento do \appName. O projeto apresenta testes unitários, de integração e features. O uso de testes automatizados facilitou as mudanças no código no decorrer do projeto, de forma a garantir o funcionamento dos códigos mais antigos sem muito esforço adicional.

\subsection{Visual Studio Code}

Visual Studio Code\footnote{\url{https://code.visualstudio.com/}} é um editor de código fonte leve, mas com recursos poderosos, disponível gratuitamente para os sistemas operacionais Windows, Linux e macOS. A ferramenta \textit{open-source} tem suporte \textit{built-in} para JavaScript, TypeScript e Node.js e possui um ecossistema rico de extensões para outras linguagens e ambientes de execução \cite{vscode}.

A escolha do editor foi feita por preferência pessoal dos autores do \appName, sendo os principais fatores decisivos, a familiaridade com a ferramenta, o suporte nativo para Typescript e React e a gratuidade. O Visual Studio Code também inclui o \textit{Intellisense}, que é uma ferramenta que provê \textit{auto-complete} inteligente para o código, baseando-se em análises semânticas da linguagem usada e na análise do código-fonte escrito \cite{intellisense}. O Visual Studio Code juntamente com o \textit{Intellisense} permitiu uma maior produtividade durante todo o processo de desenvolvimento do aplicativo.

\subsection{Firebase}
\label{firebase}

Firebase\footnote{\url{https://firebase.google.com/}} é uma plataforma de desenvolvimento de aplicações que auxilia na criação e no crescimento delas \cite{firebase}. A plataforma pertence ao Google e é gratuita de forma limitada, sendo necessário aderir a um plano pago para continuar utilizando após atingir uma quota. Dentre os serviços oferecidos, podem ser encontrados serviços de autenticação e gerenciamento de usuários e banco de dados. A plataforma adere à categoria de serviço \textit{low-code}, onde o programador tem acesso a um sistema de \textit{back-end} utilizando poucas linhas de código, sendo necessário apenas criar a aplicação que irá consumir os serviços.

O aplicativo desenvolvido utiliza o serviço de autenticação do Firebase e também o banco de dados da plataforma, denominado \textit{Firestore}, que é um banco de dados NoSQL em núvem flexível e escalável para armazenar e sincronizar dados para o desenvolvimento do lado do cliente e do servidor \cite{firestore}. A escolha da ferramenta se deu pela flexibilidade e as boas condições do plano gratuito, que atenderam todas as necessidades do projeto. Por se tratar de um banco de dados NoSQL, o banco de dados utilizado, não foram produzidos artefatos adicionais de documentação para demonstrar a forma como os dados estão estruturados e como são relacionados, pois o diagrama de classes na \refFig{class_diagram} já contém todas as informações necessárias para a modelagem do banco.
