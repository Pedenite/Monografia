O desenvolvimento do aplicativo móvel \appName\ foi feito utilizando diversas tecnologias e ferramentas, assim como metodologias. A seguir neste capítulo, serão descritas cada uma das etapas do processo de criação do \textit{app}.

\section{Processo}

Inicialmente, para o desenvolvimento do \appName, foi criado um protótipo de telas do aplicativo móvel, o que facilitaria a criação das telas funcionais posteriormente. Durante esta etapa, o estilo do aplicativo foi definido, incluindo as cores predominantes, diferentes telas pertencentes à interface de usuário e as funções acessíveis por cada tela. Tudo isso conferiu uma identidade ao projeto de software.

Com o protótipo criado e aprovado, foi utilizado um \textit{software} de organização de tarefas em times para realizar o cadastro dos itens a serem implementados no aplicativo. Com a separação dos itens entre a dupla, foi iniciado o desenvolvimento do aplicativo em questão. Durante o desenvolvimento, foram usados conceitos de metodologias ágeis para agilizar o processo, incluindo reuniões diárias e quadro kanban.

\section{Figma}

Para a criação dos protótipos, foi utilizado o \textit{software} Figma de forma gratuita. Trata-se de uma plataforma para design hospedada em formato de aplicação \textit{Web} que permite a criação de conta para o uso gratuito, porém também possuindo planos pagos de assinatura. Com as ferramentas disponibilizadas pelo figma, foram prototipadas todas as telas do aplicativo, assim como o fluxo de navegação por parte do usuário final. O protótipo pode ser dividido em 4 partes:

\subsection{Autenticação}

Composta por 3 telas: Login, Registro e \textit{Reset} de Senha. As telas são navegáveis entre si por meio dos \textit{links} clicáveis de cor azul ou cinza.

\figura{../img/figma_auth.png}{Protótipos de telas pertencentes às rotas de autenticação}{figma_auth}{width=0.75\textwidth}

\subsection{Cursos}

Definitivamente a parte mais importante do aplicativo. Foram prototipadas 6 telas para este item, incluindo a tela inicial do aplicativo após efetuar o \textit{login}.

\figura{../img/figma_course1.png}{Protótipos de telas pertencentes às rotas de autenticação}{figma_course1}{width=0.75\textwidth}

\subsection{Gerenciamento de Conta}

Também composta por 3 telas, uma para visualização dos dados de usuário, com opções para alteração dos dados e para fazer o \textit{Logout}, enquanto as outras são para realizar a alteração efetiva dos dados pessoais e senha.

\figura{../img/figma_account.png}{Protótipos de telas pertencentes às rotas de gerenciamento de conta}{figma_account}{width=0.75\textwidth}
