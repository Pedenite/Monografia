O desenvolvimento do aplicativo móvel \appName\ foi feito utilizando diversas tecnologias e ferramentas, assim como metodologias. A seguir neste capítulo, serão descritas cada uma das etapas do processo de criação do \textit{app}.

\section{Processo}

O processo de \textit{software} é visto por uma sequência de atividades
que produzem uma variedade de documentos, resultando em um programa
satisfatório e executável \cite{processos_software}. Cada vez mais, os processos se mostram tão importantes quanto o \textit{software} em si, sendo indispensáveis para a produção de um produto de maior qualidade. O desenvolvimento da aplicação foi feito utilizando metodologias ágeis, aplicando conceitos do Scrum.

\cite{agile_software_development}

Todos os recursos e programas utilizados durante qualquer etapa do projeto \appName\ foram de licença livre ou de uso gratuito. Inicialmente, para o desenvolvimento do aplicativo móvel, foram criados protótipos de telas do projeto, o que facilitaria a criação das telas funcionais posteriormente. Durante esta etapa, o estilo do aplicativo foi definido, incluindo as cores predominantes, diferentes telas pertencentes à interface de usuário, rotas de navegação e as funções acessíveis por cada uma das telas projetadas. Tudo isso conferiu uma identidade ao projeto de \textit{software}.

TODO: referência sobre a psicologia das cores.

Com o protótipo inicial criado e aprovado, foi utilizado um \textit{software} de organização de tarefas para realizar o cadastro dos itens a serem implementados no aplicativo. Com a separação dos itens entre a dupla, foi iniciado o desenvolvimento do aplicativo em questão, sendo usados conceitos de metodologias ágeis, incluindo reuniões diárias e quadro kanban.

\subsection{Figma}

TODO: referência sobre a importância dos protótipos em um projeto de software.

Para a criação dos protótipos, foi utilizado o \textit{software} Figma de forma gratuita. Trata-se de uma plataforma para design hospedada em formato de aplicação \textit{Web} que permite a criação de conta para o uso gratuito, porém também possuindo planos pagos de assinatura. Com as ferramentas disponibilizadas pelo figma, foram prototipadas todas as telas do aplicativo de acordo com a ideia inicial do projeto, assim como o fluxo de navegação por parte do usuário final. O protótipo pode ser dividido em 4 partes:

\subsubsection{Autenticação}

Composta por 3 telas: Login, Registro e \textit{Reset} de Senha. As telas são navegáveis entre si por meio dos \textit{links} clicáveis de cor azul ou cinza.

\figura[H]{../img/figma_auth.png}{Protótipos de telas pertencentes às rotas de autenticação}{figma_auth}{width=0.75\textwidth}

\subsubsection{Cursos}

Definitivamente a parte mais importante do aplicativo, sendo composta por \refFig{figma_course1} e \refFig{figma_course2}. Foram prototipadas 6 telas para este item, incluindo a tela inicial chamada de \textit{Pré-Home}, onde o usuário é redirecionado após efetuar o \textit{login}. A tela \textit{Home} é carregada por meio do botão para acessar a área de cada curso, onde se encontram os principais dados do curso, como os tópicos e o \textit{ranking}. Ao acessar um tópico a partir da tela \textit{Home}, o usuário é redirecionado para a tela deste tópico, onde pode selecionar um subtópico para resolver desafios ou visualizar o \textit{ranking} local.

\figura[H]{../img/figma_course1.png}{Protótipos de telas contendo a \textit{Home} do aplicativo e seleção de Tópicos}{figma_course1}{width=0.75\textwidth}

Pela \refFig{figma_course2}, é possível visualizar a tela do subtópico com as opções de acessar os desafios disponíveis e o ranking. A tela de desafio contém um enunciado com uma imagem opcional e campos para resposta. Por fim, a tela de classificação vai conter os usuários ordenados pelo melhor aproveitamento do curso.

\figura[H]{../img/figma_course2.png}{Protótipos de telas contendo a página de tópico com opções para nevegação para os desafios cadastrados e para o ranking}{figma_course2}{width=0.75\textwidth}

\subsubsection{\textit{Chat}}

Acessível pela barra de navegação, a tela de \textit{Chats} contém todas as pessoas com as quais o usuário logado se comunicou pelo aplicativo. Conforme mostrado na \refFig{figma_chat}, a tela de conversa é carregada ao selecionar um dos usuários listados.

\figura[H]{../img/figma_chat.png}{Protótipos de telas contendo as funções de envio e recebimanto de mensagens no \appName}{figma_chat}{width=0.5\textwidth}

\subsubsection{Gerenciamento de Conta}

Rota composta por 3 telas mostradas na \refFig{figma_account}, acessíveis pela barra de navegação, sendo uma para visualização dos dados de usuário, com opções para alteração dos dados e para fazer o \textit{Logout}, enquanto as outras são para realizar a alteração efetiva dos dados pessoais e senha.

\figura[H]{../img/figma_account.png}{Protótipos de telas pertencentes às rotas de gerenciamento de conta}{figma_account}{width=0.75\textwidth}

\subsection{\textit{Lucid Charts}}

TODO: referência da importância de fluxogramas

\textit{Lucid Charts} é uma ferramenta \textit{online} com plano gratuito para realizar o desenho de fluxos, podendo ser feito de forma colaborativa. Com as ferramentas disponibilizadas pela plataforma, foi criado o fluxograma da \refFig{fluxo_app}, que mostra as funcionalidades acessíveis pelo usuário final por cada tela. Também é mostrado o funcionamento básico dos desafios incluídos na plataforma, que fornecem uma pontuação ao usuário somente quando a pergunta é respondida corretamente.

\figura[H]{../img/fluxo_app.png}{Fluxo de utilização do \appName\ na visão do usuário}{fluxo_app}{width=\textwidth}

Pela \refFig{fluxo_app}, é possível visualizar o início e o fim do fluxo de usabilidade do aplicativo em cor vermelha, sendo o início, o \textit{login} e o fim, o \textit{logout}. As telas foram represantadas por retângulos de cor azul e as condições por losangos de cor azul claro. Por fim, funcionalidades específicas, como a forma de incremento de ponto do desafio é representada pelo pentágono irregular de cor amarelada.

\subsection{Notion}

TODO: referência sobre metodologias ágeis e kanban.

O software \textit{Notion} é uma plataforma de produtividade utilizada principalmente para organizar trabalhos em equipe. A ferramenta dispõe de diversos recursos relacionados a metodologias ágeis, documentação, métricas, entre outros, porém para o \appName, foi utilizado apenas o quadro kanban para a divisão e organização das tarefas do projeto.

\subsection{Git e Git Flow}

TODO: referências de git

Durante todo o desenvolvimento do projeto, foi utilizada a ferramenta Git para o controle de versão do código fonte. Para o compartilhamento do código atual e diferentes versões sentre a dupla, foi usado o GitHub como repositório remoto. A plataforma é de uso livre e funciona no formato \textit{Web}, facilitando o uso do Git, assim como acrescentando funcionalidades de organização de projetos que ajudam muito nos processos de desenvolvimento.

Adicionalmente, para aprimorar o uso efetivo do Git, foi utilizado um fluxo de criação de \textit{branches} intitulado \textit{Git Flow}, que consiste de regras de nomenclatura e de hierarquia para as ramificações criadas em um projeto. Na \refFig{gitflow} é possível visualizar as nomenclaturas das \textit{branches}, assim como de onde são feitos os \textit{checkouts} e \textit{merges} para cada uma.

\figuraBib[H]{../img/gitflow.png}{Exemplo visual do Git Flow}{gitflow_atlassian}{gitflow}{width=.75\textwidth}

TODO: referências sobre gitflow em projetos.

O uso do gitflow permitiu uma organização melhor do projeto para a realização do desenvolvimento de diversas funcionalidades simultaneamente sem interferência de uma nas outras. Com o uso dos \textit{Pull requests} do Github, foi possível aprimorar ainda mais o processo de \textit{merge} com a \textit{branch Develop} após o desenvolvimento de funcionalidades incluindo o \textit{Code Review} para a aprovação do item, permitindo uma melhor integração da dupla e conhecimento sobre o \textit{status} do projeto.

\section{Tecnologias Utilizadas}

Para o desenvolvimento do projeto de \textit{software}, diversas tecnologias foram utilizadas, incluindo linguagens de programação, \textit{frameworks}, ferramentas e ambientes. Nas seções seguintes, estão descritas cada uma delas.

\subsection{Android}

Android é o sistema operacional para \textit{smartphones} mais utilizado na atualidade, possuindo uma quota de mercado de 71.96\% segundo dados de Novembro de 2022 \cite{mobile-os}. Trata-se de um sistema \textit{open-source} baseado em Linux composto por uma pilha criada para suportar uma vasta quantidade de dispositivos e aspectos dos mesmos \cite{android}. A hierarquia do sistema android pode ser vista da seguinte forma:

\begin{itemize}
    \item \textbf{Aplicativos do sistema}: Todos os aplicativos que interagem diretamente com o usuário como calculadora, câmera, e-mail, entre outros;
    \item \textbf{Java API Framework}: Provedores de conteúdo, sistema de views e gerenciadores responsáveis por manter os aplicativos funcionando para o usuário final;
    \item \textbf{Bibliotecas C/C++ nativas e Runtime do Android}: Bibliotecas e funcionalidades requeridas pelos componentes e serviços do Android como o OpenGL e \textit{Android Core Libraries};
    \item \textbf{Hardware Abstract Layer (HAL)}: Camada de abstração de \textit{hardware}, é responsável por garantir o correto funcionamento das funcionalidades disponibilizadas pelo Android por meio de qualquer equipamento equipado ao aparelho como câmera, microfone, saída de audio, \textit{bluetooth}, entre outros.
    \item \textbf{Linux Kernel}: A fundação de todo o sistema Android, fornece funcionalidades de segurança, concorrência, gerenciamento de memória, entre muitos outros recursos de baixo nível para todas as camadas acima. Por ser um produto que já está no mercado a algumas décadas, também facilita para os fabricantes desenvolverem aparelhos para um \textit{Kernel} mais conhecido.
\end{itemize}

Todo o código do aplicativo \appName\ foi testado no sistema Android, porém, devido a utilização do React Native (\ref{react-section}) no código-fonte, o aplicativo é capaz de executar em dispositivos iOS também. Devido a falta de testes para a plataforma do iOS, não há garantias do correto funcionamento de todas as funcionalidades, ao contrário da versão para sistemas Android.

\subsection{Typescript}

Typescript é uma linguagem de programação \textit{open-source} fortemente tipada que foi desenvolvida e é mantida pela Microsoft e é construída em cima da linguagem Javascript, permitindo um melhor uso ferramental a qualquer escala \cite{typescript}. Por sua vez, Javascript é uma linguagem leve, interpretada ou compilada \textit{just-in-time} com funções de primeira classe. É uma linguagem multi-paradigma que é usada vastamente em páginas da \textit{Web}, mas também tem ganhado uma popularidade explosiva em sistemas \textit{back-end} e \textit{Mobile} com tecnologias como Node.js e React Native \cite{javascript}.

A escolha da linguagem se deu por razões pessoais como a familiaridade com a linguagem e devido às opções que o typescript proporciona em relação à tipagem, que têm a capacidade de aumentar a produtividade ao se utilizar editores de texto avançados, assim como diminuir a quantidade de erros em tempo de execução

\subsection{React e React Native}
\label{react-section}

React é uma biblioteca Javascript para realizar a criação de interfaces de usuário. Trata-se de um projeto \textit{open-source} criado e mantido pela empresa Meta. A ideia geral da biblioteca é ser declarativa, baseada em componentes e multi-plataforma, permitindo a criação de um código modularizado mais simples de debugar, e que pode ser convertido para outras plataformas de forma relativamente fácil \cite{react}.

A sintaxe do React utiliza a extensão de sintaxe do Javascript chamada JSX, que se assemelha muito com a sintaxe de linguagens de marcação de texto como HTML e XML, porém traduzindo a sintaxe para códigos Javascript. Por este motivo, a biblioteca pode ser aprendida com uma certa facilidade por desenvolvedores que já estejam familiarizados com o desenvolvimento web nativo usando HTML.

React Native compartilha dos mesmos princípios, funcionalidades e sintaxe do React para o desenvolvimento de aplicativos nativos. Os componentes presentes no código escrito em react native são transformados em elementos de interfaces de usuário utilizando a API da própria plataforma onde será executado. O \textit{framework} facilita o desenvolvimento para plataformas móveis de forma a utilizar o mesmo código tanto no Android como no iOS, não exigindo muito retrabalho ao criar suporte a outras plataformas para o aplicativo. Assim como React, o React Native é uma biblioteca \textit{open-source} mantida pela Meta \cite{react-native}.

A escolha das ferramentas para o projeto \appName foi devido à familiaridade com a mesma por parte da dupla, não exigindo um treinamento extensivo para seu uso no projeto. A aplicação móvel, chamada de \textit{front-office} foi desenvolvida usando React Native, enquanto a versão \textit{Web}, \textit{back-office}, foi desenvolvida utilizando React.

\subsection{Visual Studio Code}

Visual Studio Code é um editor de código fonte leve, mas com recursos poderosos, disponível gratuitamente para os sistemas operacionais Windows, Linux e macOS. A ferramenta \textit{open-source} tem suporte \textit{built-in} para JavaScript, TypeScript e Node.js e possui um ecossistema rico de extensões para outras linguagens e ambientes de execução \cite{vscode}.

A escolha do editor foi feita por preferência pessoal dos autores do \appName, sendo os principais fatores decisivos, a familiaridade com a ferramenta, o suporte nativo para Typescript e React e a gratuidade. O Visual Studio Code também inclui o \textit{Intellisense}, que é uma ferramenta que provê \textit{auto-complete} inteligente para o código, baseando-se em análises semânticas da linguagem usada e na análise do código-fonte escrito \cite{intellisense}. O Visual Studio Code juntamente com o Intellisense permitiu uma maior produtividade durante todos o processo de desenvolvimento do aplicativo.

\subsection{Firebase}

Firebase \cite{firebase} é uma plataforma de desenvolvimento de aplicações que auxilia na criação e no crescimento delas. A plataforma pertence ao Google e é gratuita de forma limitada, sendo necessário aderir a um plano pago para continuar utilizando após atingir uma quota. Dentre os serviços oferecidos, podem ser encontrados serviços de autenticação e gerenciamento de usuários e banco de dados. A plataforma adere à categoria de serviço \textit{low-code}, onde o programador tem acesso a um sistema de \textit{backend} utilizando poucas linhas de código, sendo necessário apenas criar a aplicação que irá consumir os serviços.

O aplicativo desenvolvido utiliza o serviço de autenticação do Firebase e também o banco de dados da plataforma, denominado \textit{Firestore}, que é um banco de dados NoSQL em núvem flexível e escalável para armazenar e sincronizar dados para o desenvolvimento do lado do cliente e do servidor \cite{firestore}. A escolha da ferramenta se deu pela flexibilidade e as boas condições do plano gratuito, que atenderam todas as necessidades do projeto.
