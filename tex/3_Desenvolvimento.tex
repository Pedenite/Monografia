O desenvolvimento do aplicativo móvel \appName\ foi feito utilizando diversas tecnologias e ferramentas, assim como metodologias. A seguir neste capítulo, serão descritas cada uma das etapas do processo de criação do \textit{app}.

\section{Processo}

TODO: referências\dots

Todos os recursos e programas utilizados durante qualquer etapa do projeto \appName\ foram de licença livre ou de uso gratuito. Inicialmente, para o desenvolvimento do aplicativo móvel, foram criados protótipos de telas do projeto, o que facilitaria a criação das telas funcionais posteriormente. Durante esta etapa, o estilo do aplicativo foi definido, incluindo as cores predominantes, diferentes telas pertencentes à interface de usuário, rotas de navegação e as funções acessíveis por cada uma das telas projetadas. Tudo isso conferiu uma identidade ao projeto de software.

TODO: referência sobre a psicologia das cores.

Com o protótipo criado e aprovado, foi utilizado um \textit{software} de organização de tarefas para realizar o cadastro dos itens a serem implementados no aplicativo. Com a separação dos itens entre a dupla, foi iniciado o desenvolvimento do aplicativo em questão, sendo usados conceitos de metodologias ágeis, incluindo reuniões diárias e quadro kanban.

\subsection{Figma}

TODO: referência sobre a importância dos protótipos em um projeto de software.

Para a criação dos protótipos, foi utilizado o \textit{software} Figma de forma gratuita. Trata-se de uma plataforma para design hospedada em formato de aplicação \textit{Web} que permite a criação de conta para o uso gratuito, porém também possuindo planos pagos de assinatura. Com as ferramentas disponibilizadas pelo figma, foram prototipadas todas as telas do aplicativo de acordo com a ideia inicial do projeto, assim como o fluxo de navegação por parte do usuário final. O protótipo pode ser dividido em 4 partes:

\subsubsection{Autenticação}

Composta por 3 telas: Login, Registro e \textit{Reset} de Senha. As telas são navegáveis entre si por meio dos \textit{links} clicáveis de cor azul ou cinza.

\figura[H]{../img/figma_auth.png}{Protótipos de telas pertencentes às rotas de autenticação}{figma_auth}{width=0.75\textwidth}

\subsubsection{Cursos}

Definitivamente a parte mais importante do aplicativo, sendo composta por \refFig{figma_course1} e \refFig{figma_course2}. Foram prototipadas 6 telas para este item, incluindo a tela inicial chamada de \textit{Pré-Home}, onde o usuário é redirecionado após efetuar o \textit{login}. A tela \textit{Home} é carregada por meio do botão para acessar a área de cada curso, onde se encontram os principais dados do curso, como os tópicos e o \textit{ranking}. Ao acessar um tópico a partir da tela \textit{Home}, o usuário é redirecionado para a tela deste tópico, onde pode selecionar um subtópico para resolver desafios ou visualizar o \textit{ranking} local.

\figura[H]{../img/figma_course1.png}{Protótipos de telas contendo a \textit{Home} do aplicativo e seleção de Tópicos}{figma_course1}{width=0.75\textwidth}

Pela \refFig{figma_course2}, é possível visualizar a tela do subtópico com as opções de acessar os desafios disponíveis e o ranking. A tela de desafio contém um enunciado com uma imagem opcional e campos para resposta. Por fim, a tela de classificação vai conter os usuários ordenados pelo melhor aproveitamento do curso.

\figura[H]{../img/figma_course2.png}{Protótipos de telas contendo a página de tópico com opções para nevegação para os desafios cadastrados e para o ranking}{figma_course2}{width=0.75\textwidth}

\subsubsection{\textit{Chat}}

Acessível pela barra de navegação, a tela de \textit{Chats} contém todas as pessoas com as quais o usuário logado se comunicou pelo aplicativo. Conforme mostrado na \refFig{figma_chat}, a tela de conversa é carregada ao selecionar um dos usuários listados.

\figura[H]{../img/figma_chat.png}{Protótipos de telas contendo as funções de envio e recebimanto de mensagens no \appName}{figma_chat}{width=0.5\textwidth}

\subsubsection{Gerenciamento de Conta}

Rota composta por 3 telas mostradas na \refFig{figma_account}, acessíveis pela barra de navegação, sendo uma para visualização dos dados de usuário, com opções para alteração dos dados e para fazer o \textit{Logout}, enquanto as outras são para realizar a alteração efetiva dos dados pessoais e senha.

\figura[H]{../img/figma_account.png}{Protótipos de telas pertencentes às rotas de gerenciamento de conta}{figma_account}{width=0.75\textwidth}

\subsection{\textit{Lucid Charts}}

TODO: referência da importância de fluxogramas

\textit{Lucid Charts} é uma ferramenta \textit{online} com plano gratuito para realizar o desenho de fluxos, podendo ser feito de forma colaborativa. Com as ferramentas disponibilizadas pela plataforma, foi criado o fluxograma da \refFig{fluxo_app}, que mostra as funcionalidades acessíveis pelo usuário final por cada tela. Também é mostrado o funcionamento básico dos desafios incluídos na plataforma, que fornecem uma pontuação ao usuário somente quando a pergunta é respondida corretamente.

\figura[H]{../img/fluxo_app.png}{Fluxo de utilização do \appName\ na visão do usuário}{fluxo_app}{width=\textwidth}

Pela \refFig{fluxo_app}, é possível visualizar o início e o fim do fluxo de usabilidade do aplicativo em cor vermelha, sendo o início, o \textit{login} e o fim, o \textit{logout}. As telas foram represantadas por retângulos de cor azul e as condições por losangos de cor azul claro. Por fim, funcionalidades específicas, como a forma de incremento de ponto do desafio é representada pelo pentágono irregular de cor amarelada.

\subsection{Notion}

TODO: referência sobre metodologias ágeis e kanban.

O software \textit{Notion} é uma plataforma de produtividade utilizada principalmente para organizar trabalhos em equipe. A ferramenta dispõe de diversos recursos relacionados a metodologias ágeis, documentação, métricas, entre outros, porém para o \appName, foi utilizado apenas o quadro kanban para a divisão e organização das tarefas do projeto.

\section{Tecnologias Utilizadas}

TODO\dots

\subsection{Android}

TODO\dots

\subsection{Typescript}

TODO\dots

\subsection{React e React Native}

TODO\dots

\subsection{Firebase}

TODO\dots
