A gamificação \cite{gamification_motivates} é o processo de implementação de elementos de \textit{design} de jogos em contextos do mundo real para propósitos diversos, fora do contexto dos jogos em si. O principal propósito dela é para gerar uma maior motivação nas pessoas ao executarem determinada atividade. Neste capítulo, serão definidos conceitos necessários para a gamificação e serão exploradas as aplicações da mesma no processo de ensino e aprendizagem e como ela foi aplicada no projeto \appName.

\section{Jogos}

Jogo é uma expressão que define diversos conceitos diferentes, onde o significado depende de elementos como faixa etária dos participantes, local, contexto, entre muitos outros fatores \cite{jogos}. Pesquisadores do \textit{Laboratoire de Recherche sur le Jeu et le Jouet}, da \textit{Université Paris-Nord} como Gilles Brougère \cite{gilles_jogos} e Jacques Henriot \cite{jacques_jeu} endereçaram a definição do termo jogo por meio de três níveis de diferenciações, sendo eles:

\begin{enumerate}
    \item O resultado de um sistema linguístico que funciona dentro de um contexto social;
    \item Um sistema de regras;
    \item Um objeto.
\end{enumerate}

A definição acima se dá devido à mistura que é feita entre jogos, brincadeiras e brinquedos, que são tratados como um só na maioria dos contextos. Com os níveis definidos, é possível classificar os jogos, assim como diferenciá-los, pois mesmo se dois jogos diferentes utilizam o mesmo objeto para jogar como cartas de baralho, ainda é possível distinguí-los pelas regras, por exemplo.

\section{Jogos eletrônicos}

Jogos eletrônicos, popularmente conhecidos como videogames, são formas de entretenimento que se expressam no meio virtual. São jogos que permitem a interação por meio de um dispositivo eletrônico, como computadores, consoles, \textit{smartphones} e \textit{arcades}.

Segundo dados transmitidos em 2022 da Associação de \textit{softwares} de entretenimento \cite{esa_report_2022}, bilhões de pessoas utilizam os jogos eletrônicos como forma de entretenimento diariamente na atualidade, transcendendo barreiras de idade, gênero, cultura e até distância no caso de jogos de múltiplos jogadores \textit{online}. Devido à grande presença dos vídeo-games na atualidade, este se tornou um dos mercados mais rentáveis dos últimos tempos, e com um potencial ainda maior de crescimento \cite{video-game-economics}.

%%% TODO: verify copyrights
\figuraBib[H]{../img/gaming_growth_chart.png}{Gráfico de crescimento do mercado de jogos eletrônicos entre 1971 e 2018}{wikipedia_gaming_industry}{gaming_growth}{width=\textwidth}

Pelo gráfico acima, é possível visualizar um crescimento constante dos videogames como um todo desde o ano de 1997, aproximadamente. Nos últimos anos, é notável o destaque do mercado de jogos para dispositivos móveis como \textit{smartphones}, que é o dispositivo tecnológico que tem estado cada vez mais presente na vida das pessoas.

\section{A Gamificação no processo de Ensino e Aprendizagem}

Com os avanços tecnológicos dos útimos tempos, diversos setores da sociedade se viram forçados a realizarem mudanças para a inclusão das tecnologias mais atuais para possibilitar maior acessibilidade ao público e realizar inovações. Atualmente, muitos jovens do século XXI tiveram acesso a diversas tecnologias no cotidiano durante todo ou grande parte do seu desenvolvimento cognitivo. Visto isso, as escolas e os professores são motivados a se adaptar à nova realidade dos alunos para ensinar de forma mais inclusiva, possibilitando ao outro, o acesso à informação e ao conhecimento \cite{tecnologia-professores}.

A tecnologia da comunicação tem o poder de dinamizar a sala de aula, saindo de um ambiente monótono, no qual um fala e todos escutam, para um ambiente acolhedor, dinâmico com possibilidades de discussões e debates. Os novos tempos exigem um padrão educacional que esteja voltado para o desenvolvimento de um conjunto de competências e de habilidades essenciais, a fim de que os alunos possam compreender e refletir sobre a realidade, participando e agindo no contexto de uma sociedade comprometida com o futuro (SANTOS, 2010) \cite{tecnologia-professores}.

\subsection{Elementos de \textit{Design} de Jogos que podem ser aplicados na Gamificação}
\label{gamification_elements}

Diversos autores apresentam vários aspectos diferentes para o \textit{design} completo de um jogo, porém uma seleção dos principais itens apresentados recorrentemente pelos autores são: pontuação, distintivos, \textit{ranking}, gráficos de performance, histórias significativas, avatares e colegas \cite{gamification_motivates}. A seguir, cada um dos itens serão descritos detalhadamente.

\begin{itemize}
    \item \textbf{Pontuação} é um elemento básico de uma grande quantidade de jogos e aplicações gamificadas, que representa o sucesso em cumprir  determinadas atividade dentro do ambiente do jogo, representando o progresso do jogador. Dentre os principais propósitos dos pontos, estão o de permitir um \textit{feedback} para o jogador e o de servir como uma recompensa pelo trabalho feito no ambiente do jogo;
    \item \textbf{Distintivos} são representações visuais de uma ou mais conquistas e podem ser obtidos ou coletados dentro do ambiente do jogo. Têm como função, confirmar a conquista do jogador, simbolizando o mérito;
    \item \textbf{\textit{Ranking}} é uma classificação de jogadores que é feita com base no seu sucesso relativo, sendo medido de acordo com algum critério determinado pelo jogo, como por exemplo, a própria pontuação. Os efeitos motivadores de um sistema de classificação podem variar bastante, podendo causar uma grande motivação em pessoas que estão a poucos passos de passar de posição, mas também servindo como um desmotivador para os jogadores que se encontram no fundo da classificação \cite{ranking_motivation}. Apesar disso, a competição causada por um ranqueamento, pode criar uma pressão social para aumentar o nível de engajamento do jogador e, consequentemente, ter um efeito construtivo na participação e resultado \cite{ranking_competition};
    \item \textbf{Gráficos de Performance} são mais usados em jogos de simulação ou estratégia e permitem aos jogadores, o acesso à informações sobre a sua evolução no jogo, comparando a performance da partida atual com as anteriores. Ao contrário de um \textit{ranking}, o gráfico de performance não compara a perormance do jogador com outros jogadores, mas sim avalia a evolução de sua performance ao longo do tempo;
    \item \textbf{Histórias Significativas} não se relacionam com a performance do jogador. O contexto narrativo em que uma aplicação gamificada pode ser incorporada, contextualiza as atividades e características do jogo, aplicando um significado que vai além da tarefa de obter pontos e conquistas, o que pode gerar resultado positivos para a inspiração e engajamento dos jogadores, principalmente se a história estiver de acordo com seus interesses pessoais \cite{meaningful-gamification};
    \item \textbf{Avatares} são representações visuais do jogador dentro do ambiente do jogo, que podem ser escolhidos ou criados pelo jogador. Avatares permitem que os jogadores adotem ou criem uma nova identidade dentro do jogo;
    \item \textbf{Colegas} podem introduzir no jogo, o conflito, a competição e a cooperação \cite{kapp_gamification}. A cooperação pode ser obtida por meio da introdução de times, ao definir grupos de jogadores que trabalham em conjunto para atingir um determinado objetivo mútuo \cite{ranking_motivation}.
\end{itemize}

\section{O Ensino de Computação}

A área de Tecnologia da Informação e Comunicação (TIC) é um dos setores que mais tem crescido nos últimos anos em relação à demanda, oportunidades e, consequentemente, procura por qualificação profissional \cite{brasscom-tic}. Com isso, cursos da área de computação, em especial têm mostrado uma alta procura, porém, ao mesmo tempo, mostram uma das maiores taxas de evasão dentre as formações superiores e técnicas, apresentando uma taxa de evasão média anual de 51,46\% entre 2015 e 2019 para o curso de sistemas para internet analisado do Instituto Federal do Tocantins e uma taxa de 27,08\% para o curso de ciência da computação ofertado pela Universidade Federal do Tocantins \cite{evasao-computacao}.

%%%% TODO: Evasão UnB

Os motivos para a evasão escolar podem ser diversos, sendo classificados como internos ou externos. Cabe à instituição de ensino tomar as medidas necessárias para diminuir as altas taxas de evasão ao analizar os motivos para chegar em uma solução. Todavia, um método de ensino mais inclusivo poderia ter a capacidade de manter os alunos engajados no curso e promover um melhor aprendizado.

Com isso em mente, o \appName\ foi desenvolvido, utilizando dos conceitos de gamificação para auxiliar no processo de aprendizagem. As ferramentas para o uso da gamificação são disponibilizadas para que o professor que decide usar a plataforma seja capaz de montar o seu curso e matricular os alunos, assim como acompanhar os seus desempenhos. O detalhamento destas funcionalidades será descrito a seguir.

\section{Gamificação no \appName}

A maioria dos elementos de gamificação presentes na compilação apresentada na Seção \ref{gamification_elements} foram aplicados de alguma forma no aplicativo \appName. Porém, por se tratar de uma plataforma onde um professor pode criar um curso, alguns dos fatores para uma aplicação mais efetiva da gamificação dependem do conhecimento do professor sobre o tema e como ele irá aplicar o uso de conceitos de jogos nos exercícios propostos. A seguir, está descrito como foi aplicado cada um dos aspectos existentes no aplicativo sobre o que se refere à gamificação.

\begin{itemize}
    \item A \textbf{Pontuação} é individual para cada aluno e é diferente para cada time ou curso cadastrado na plataforma. Deste modo, o sistema de ranqueamento de cada time não sofrerá interferência dos outros, garantindo uma plataforma individual para cada um. O usuário ganha pontos ao resolver corretamente os desafios propostos, mas não obtém ponto algum ao responder incorretamente o desafio.
    \item O \textbf{\textit{Ranking}} foi implementado individualmente para cada curso ou time cadastrado na plataforma conforme mencionado anteriormente. A classificação é gerada unicamente com base na quantidade de pontos obtidos por cada aluno. A quantidade de pontos obtidas em cada desafio é definida pelo professor, porém é limitada a um \textit{range} para não gerar diferenças muito grandes de pontuação em cada desafio.
    \item A \textbf{Performance} dos alunos pode ser visualizada pelos professores para localizar possíveis dificuldades generalizadas ou individuais. O aluno também pode visualizar a sua performance ao finalizar os desafios de um tópico, podendo acessar um \textit{feedback} individual para cada questão ou as estatísticas gerais dos desafios realizados como a porcentagem de acerto e erro e a pontuação total obtida. Ao visualizar os seus erros e acertos, o estudante pode identificar as suas próprias dificuldades, o que permite o auto-aperfeiçoamento ao estudar mais o que lhe falta, assim como pode também recorrer à ajuda dos outros para um crescimento mútuo.
    \item \textbf{Histórias Significativas} não foram implementadas diretamente no aplicativo pois se trata de algo individual para cada contexto e, portanto, cabe ao professor a decisão de incluir a história ou não. O professor poderá seguir a sua pŕopria metodologia de ensino para despertar o interesse e engajamento dos alunos.
    \item Os \textbf{Colegas} de um determinado curso aparecem no \textit{ranking} da plataforma, permitindo que os alunos se comuniquem entre si por meio da ferramenta de conversa por mensagens implementada para solicitar ajuda ou interagir da forma que bem entenderem.
\end{itemize}

O capítulo seguinte discute aspectos mais técnicos sobre o desenvolvimento e o funcionamento do aplicativo \appName.
