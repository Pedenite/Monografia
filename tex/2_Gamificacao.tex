A gamificação é o processo de implementação de elementos de \textit{design} de jogos em contextos do mundo real para propósitos diversos, fora do contexto dos jogos em si. O principal propósito dela é para gerar uma maior motivação nas pessoas ao executarem determinada atividade \cite{gamification_motivates}. Neste capítulo, serão exploradas as aplicações da gamificação no processo de ensino e aprendizagem e como ela foi aplicada no projeto \appName.

\section{Jogos}

Jogo é uma expressão que define diversos conceitos diferentes, onde o significado depende de elementos como faixa etária dos participantes, local, contexto, entre muitos outros fatores \cite{jogos}. Pesquisadores do \textit{Laboratoire de Recherche sur le Jeu et le Jouet}, da \textit{Université Paris-Nord} como Gilles Brougère \cite{gilles_jogos} e Jacques Henriot \cite{jacques_jeu} endereçaram a definição do termo jogo por meio de três níveis de diferenciações, sendo eles:

\begin{enumerate}
    \item O resultado de um sistema linguístico que funciona dentro de um contexto social;
    \item Um sistema de regras;
    \item Um objeto.
\end{enumerate}

\section{Jogos eletrônicos}

Jogos eletrônicos, popularmente conhecidos como videogames, são formas de entretenimento que se expressam no meio virtual. São jogos que permitem a interação por meio de um dispositivo eletrônico, como computadores, consoles, \textit{smartphones} e \textit{arcades}.

Segundo dados transmitidos em 2022 da Associação de \textit{softwares} de entretenimento \cite{esa_report_2022}, bilhões de pessoas utilizam os jogos eletrônicos como forma de entretenimento diariamente na atualidade, transcendendo barreiras de idade, gênero, cultura e até distância no caso de jogos de múltiplos jogadores \textit{online}. Devido à grande presença dos vídeo-games na atualidade, este se tornou um dos mercados mais rentáveis dos últimos tempos, e com um potencial ainda maior de crescimento \cite{video-game-economics}.

\figuraBib[H]{../img/gaming_growth_chart.png}{Gráfico de crescimento do mercado de jogos eletrônicos entre 1971 e 2018}{wikipedia_gaming_industry}{gaming_growth}{width=\textwidth}

Pelo gráfico acima, é possível visualizar um crescimento constante dos videogames como um todo desde o ano de 1997, aproximadamente. Nos últimos anos, é notável o destaque do mercado de jogos para dispositivos móveis como \textit{smartphones}, que é o dispositivo tecnológico que tem estado cada vez mais presente na vida das pessoas.

\section{A Gamificação no processo de Ensino e Aprendizagem}

Com os avanços tecnológicos dos útimos tempos, diversos setores da sociedade se viram forçados a realizarem mudanças para a inclusão das tecnologias mais atuais para possibilitar maior acessibilidade ao público e realizar inovações. Atualmente, muitos jovens do século XXI tiveram acesso a diversas tecnologias no cotidiano durante todo ou grande parte do seu desenvolvimento. Visto isso, as escolas e os professores são motivados a se adaptar à nova realidade dos alunos para  ensinar de forma mais inclusiva, possibilitando ao outro, o acesso à informação e ao conhecimento \cite{tecnologia-professores}.

A tecnologia da comunicação tem o poder de dinamizar a sala de aula, saindo de um ambiente monótono, no qual um fala e todos escutam, para um ambiente acolhedor, dinâmico com possibilidades de discussões e debates. Os novos tempos exigem um padrão educacional que esteja voltado para o desenvolvimento de um conjunto de competências e de habilidades essenciais, a fim de que os alunos possam compreender e refletir sobre a realidade, participando e agindo no contexto de uma sociedade comprometida com o futuro (SANTOS, 2010) \cite{tecnologia-professores}.

\section{O Ensino de Computação}

A área de Tecnologia da Informação e Comunicação (TIC) é um dos setores que mais tem crescido nos últimos anos em relação à demanda, oportunidades e, consequentemente, procura por qualificação profissional \cite{brasscom-tic}. Com isso, cursos da área de computação, em especial têm mostrado uma alta procura, porém, ao mesmo tempo, mostram uma das maiores taxas de evasão dentre as formações superiores e técnicas, apresentando uma taxa de evasão média anual de 51,46\% entre 2015 e 2019 para o curso de sistemas para internet analisado do Instituto Federal do Tocantins e uma taxa de 27,08\% para o curso de ciência da computação ofertado pela Universidade Federal do Tocantins \cite{evasao-computacao}.

Os motivos para a evasão escolar podem ser diversos, sendo classificados como internos ou externos. Cabe à instituição de ensino tomar as medidas necessárias para diminuir as altas taxas de evasão ao analizar os motivos para chegar em uma solução. Todavia, um método de ensino mais inclusivo teria uma boa capacidade de manter os alunos engajados no curso e promover um melhor aprendizado.

Com isso em mente, o \appName\ foi desenvolvido, utilizando dos conceitos de gamificação para auxiliar no processo de aprendizagem. As ferramentas para o uso da gamificação são disponibilizadas para o professor que decide usar a plataforma montar o seu curso e matricular os alunos. O detalhamento destas funcionalidades será descrito a seguir.

\section{Gamificação no \appName}

TODO
