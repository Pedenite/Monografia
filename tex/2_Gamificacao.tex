A gamificação é o processo de implementação de elementos de \textit{design} de jogos em contextos do mundo real para propósitos diversos, fora do contexto dos jogos em si. O principal propósito dela é para gerar uma maior motivação nas pessoas ao executarem determinada atividade \cite{gamification_motivates}. Neste capítulo, serão exploradas as aplicações da gamificação no processo de ensino e aprendizagem e como ela foi aplicada no projeto \appName.

\section{Jogos}

Jogo é uma expressão que define diversos conceitos diferentes, onde o significado depende de elementos como faixa etária dos participantes, local, contexto, entre muitos outros fatores \cite{jogos}. Pesquisadores do \textit{Laboratoire de Recherche sur le Jeu et le Jouet}, da \textit{Université Paris-Nord} como Gilles Brougère \cite{gilles_jogos} e Jacques Henriot \cite{jacques_jeu} endereçaram a definição do termo jogo por meio de três níveis de diferenciações, sendo eles:

\begin{enumerate}
    \item O resultado de um sistema linguístico que funciona dentro de um contexto social;
    \item Um sistema de regras;
    \item Um objeto.
\end{enumerate}

\section{Jogos eletrônicos}

Jogos eletrônicos, popularmente conhecidos como videogames, são formas de entretenimento que se expressam no meio virtual. São jogos que permitem a interação por meio de um dispositivo eletrônico, como computadores, consoles, \textit{smartphones} e \textit{arcades}.

Segundo dados transmitidos em 2022 da Associação de \textit{softwares} de entretenimento \cite{esa_report_2022}, bilhões de pessoas utilizam os jogos eletrônicos como forma de entretenimento diariamente na atualidade, transcendendo barreiras de idade, gênero, cultura e até distância no caso de jogos de múltiplos jogadores \textit{online}. Devido à grande presença dos vídeo-games na atualidade, este se tornou um dos mercados mais rentáveis dos últimos tempos, e com um potencial ainda maior de crescimento \cite{video-game-economics}.

\figuraBib[H]{../img/gaming_growth_chart.png}{Gráfico de crescimento do mercado de jogos eletrônicos entre 1971 e 2018}{wikipedia_gaming_industry}{gaming_growth}{width=\textwidth}

Pelo gráfico acima, é possível visualizar um crescimento constante dos videogames como um todo desde o ano de 1997, aproximadamente. Nos últimos anos, é notável o destaque do mercado de jogos para dispositivos móveis como \textit{smartphones}.

\section{A Gamificação no processo de Ensino e Aprendizagem}

Com os avanços tecnológicos dos útimos tempos, diversos setores da sociedade se viram forçados a realizarem mudanças para a inclusão das tecnologias mais atuais para possibilitar maior acessibilidade ao público e realizar inovações. Atualmente, muitos jovens do século XXI tiveram acesso a diversas tecnologias no cotidiano durante todo ou grande parte do seu desenvolvimento. Visto isso, o ensino por parte das escolas e dos professores são motivados a se adaptar à nova realidade dos alunos de forma mais inclusiva, possibilitando ao outro, o acesso à informação e ao conhecimento \cite{tecnologia-professores}.

\section{O Ensino de Computação}

TODO: dados sobre evasão nos cursos de computação

\section{Gamificação no \appName}

TODO
