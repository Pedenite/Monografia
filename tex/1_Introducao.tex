Com os avanços tecnológicos observados no decorrer das últimas décadas, surgem, cada vez mais, oportunidades de melhorar os modelos existentes de ensino e aprendizagem. Nos últimos anos, a tecnologia vem se tornando uma das principais aliadas dos estudantes dentro e fora das salas de aula. Existem diversas ferramentas disponíveis para diferentes nichos educacionais, mas nem todas são gratuitas, acessíveis e, tampouco, despertam o interesse dos alunos.

Pensando nessa problemática, este presente trabalho tem como objetivo, atender os estudantes que se interessam em aprender conteúdos de computação. Para isso, foi desenvolvido um aplicativo, intitulado \appName, para dispositivos móveis, onde os estudantes poderão, gratuitamente, aprender de forma prática e teórica, diversos tópicos relacionados às áreas de seu interesse.

\appName\ usa uma abordagem \textit{gamificada} para tornar o aprendizado mais interessante e, consequentemente, mais fácil de alcançar o aluno. No aplicativo, o usuário pode acessar o tópico que deseja e realizar atividades relacionadas que são disponibilizadas em formato de questões. Cada questão respondida corretamente irá incrementar a pontuação do usuário na plataforma, contribuindo para o seu \textit{rank} geral e local (Por disciplina). O aluno, em caso de dificuldades, também pode contar com a ajuda de colegas que encontrar no \textit{ranking} que estiverem disponíveis a ajudar por meio do chat interno presente na plataforma. No \textit{ranking}, os usuários são capazes de visualizar todos os colegas e podem ver os que estão com o melhor desempenho na disciplina, sendo capazes de visitar a página do respectivo usuário onde estará disponível a opção para iniciar um \textit{chat}, assim como outras ações.

Para a progressão dos cursos, \appName\ conta com uma trilha de aprendizagem para que os estudantes possam acessar as disciplinas em uma ordem mais eficiente, de acordo com a árvore de pré-requisitos. Para usuários que já estiverem em um nível mais avançado dos cursos, uma avaliação será disponibilizada para se obter uma sugestão de curso a se fazer, porém, esta sugestão pode ser ignorada se assim o usuário desejar. Esta abordagem tem como objetivo, deixar mais consistente o uso do aplicativo na visão do usuário, que será capaz de escolher se deseja seguir o que lhe foi indicado ou seguir a sua própria rota.

Tudo o que foi descrito até o momento, mostra a visão do estudante sobre o sistema, que chamamos de \textit{front office}, porém, também existe a visão do professor sobre o \appName, que é o chamado \textit{back office}, cujo desenvolvimento está feito em formato de uma aplicação \textit{web}. Neste, o usuário, que seria o professor das disciplinas, terá acesso a ferramentas com funções mais administrativas, tais como: incluir disciplinas, matricular alunos, adicionar atividades, emitir relatórios, entre outros.
