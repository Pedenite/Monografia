Com os avanços tecnológicos observados no decorrer das últimas décadas, surgem, cada vez mais, oportunidades de melhorar os modelos existentes de ensino e aprendizagem. Nos últimos anos, a tecnologia vem se tornando uma das principais aliadas dos estudantes dentro e fora das salas de aula. Existem diversas ferramentas disponíveis para diferentes nichos educacionais, mas nem todas são gratuitas, acessíveis e, tampouco, despertam o interesse dos alunos.

Este projeto tem como objetivo, criar uma plataforma digital, intitulada \appName\, para auxiliar estudantes e professores no ensino e aprendizagem de tópicos do ramo computacional. \appName\ utiliza uma abordagem gamificada \cite{gamification_motivates} para tornar o aprendizado mais interessante e, consequentemente, mais fácil de alcançar o aluno.

\section{Problema}

Existe uma carência de ferramentas capazes de superar as barreiras para o aprendizado de temáticas computacionais. A aprendizagem de computação está comumente relacionada com longas horas de estudos em um ambiente silencioso e com bons equipamentos. Embora estes fatores possam facilitar o processo de aprendizagem, eles não devem ser responsáveis por limitar a busca por conhecimento. Entretanto, a falta de tempo, e de um ambiente adequado, tornam-se grandes impeditivos, capazes de gerar nos estudantes, frustrações ao iniciarem seus estudos sobre determinados assuntos.

\section{Justificativa}
Devido a grande necessidade da construção de recursos didáticos capazes de auxiliar estudantes na busca por conhecimento de elementos da computação, este presente trabalho tem como objetivo, atender, de forma acessível, os estudantes que desejam aperfeiçoar ou adquirir conhecimentos acerca de diversas temáticas computacionais. Para isso, foi desenvolvido uma plataforma digital, que engloba um aplicativo para dispositivos móveis gamificado, onde os estudantes poderão, gratuitamente, aprender de forma prática, diversos tópicos relacionados às áreas de seu interesse. A plataforma conta também, com uma aplicação \textit{web} para auxiliar professores em práticas pedagógicas.

\section{Objetivo Geral}
Construir uma plataforma digital para solucionar problemas relacionados ao processo de aprendizagem que os estudantes enfrentam no âmbito computacional. Além disso, a plataforma serve de auxílio para professores que desejam utilizar recursos digitais em suas práticas docentes.

\section{Objetivos Específicos}

A fim de atender os objetivos finais do projeto, foram estabelecidos os seguintes objetivos específicos:

\begin{itemize}
  \item Desenvolver um aplicativo que integre elementos de gamificação em um ambiente de ensino de computação para aumentar a motivação e o envolvimento dos alunos;
  \item Fornecer uma ferramenta para que os professores usem a tecnologia de maneira efetiva e integrem-na em sua prática docente para melhorar a aprendizagem dos alunos e tornar o ensino mais eficiente e interessante;
  \item Oferecer um mecanismo de \textit{feedback} imediato e eficaz para os alunos;
  \item Acompanhar o progresso dos alunos e fornecer relatórios detalhados aos professores.
\end{itemize}

\section{Metodologia}
Este trabalho foi desenvolvido segundo a linha de pesquisa exploratória. Foram visitadas fontes diversas a respeito das falhas no processo de ensino e aprendizagem de diversas áreas da computação, em específico, relacionadas ao desenvolvimento do pensamento computacional. As etapas da construção do projeto estão relacionadas em:

\begin{itemize}
  \item Pesquisa sobre os principais pontos de defasagem no ensino de computação;
  \item Estudo sobre a abordagem da gamificação no processo de aprendizagem;
  \item \textit{Benchmark} com as principais ferramentas de \textit{e-learning};
  \item Estabelecer critérios de aceite;
  \item Construção de protótipos, fluxogramas e documentação;
  \item Desenvolvimento das aplicações móvel e web;
  \item Validação da plataforma por meio de testes unitários, funcionais e exploratórios;
  \item Avaliar os impactos da plataforma na aprendizagem dos estudantes, e no auxílio aos professores de computação.
\end{itemize}

\section{Trabalhos Relacionados}
Para que o escopo das aplicações fosse definido, e para fins de referência, foram levantados aplicativos educacionais de diversos segmentos, utilizando-se como base as fontes de pesquisa do Play Store\footnote{\url{https://play.google.com}}, que contém os aplicativos em questão para baixar e o Google Scholar\footnote{\url{https://scholar.google.com/}}, que contém artigos acadêmicos e livros com estudos de caso sobre eles. As principais palavras-chave utilizadas como parâmetros de busca foram: \textit{Coding Apps}, \textit{E-learning}, \textit{Game-based Learning} e Jogos Educacionais. Com base nas pesquisas, os aplicativos que mais aproximaram-se do contexto do projeto são:

\begin{itemize}
  \item \textbf{Duolingo}\footnote{\url{https://pt.duolingo.com/}} \cite{duolingo}: aplicativo para ensino de línguas estrangeiras; Possui foco em gamificação; Alto engajamento popular; O diferencial do aplicativo é sua trilha de aprendizado, possibilitando o usuário monitorar o seu progresso por meio de uma interface amigável; Disponível gratuitamente em diversos idiomas, incluindo Português;
  \item \textbf{Khan Academy}\footnote{\url{https://pt.khanacademy.org}} \cite{khan_academy}: aplicativo para ensino de conteúdos diversos, em especial, conteúdos acadêmicos e de desenvolvimento pessoal; Possui elementos de gamificação; Alto engajamento popular; O diferencial do aplicativo é sua seção de ensino, com professores especializados em diversas áreas do conhecimento; Disponível gratuitamente em diversos idiomas, incluindo Português;
  \item \textbf{Beecrowd}\footnote{\url{https://www.beecrowd.com.br}}: aplicativo web para ensino de algoritmos e programação, focado em competição; Possui elementos de gamificação; Alto engajamento popular; O diferencial do aplicativo é seu intenso foco em programação competitiva, fazendo com que os estudantes se mantenham engajados na busca por um bom posicionamento no \textit{ranking}; Disponível gratuitamente em diversos idiomas, incluindo Português.
\end{itemize}

O \appName\ se diferencia das outras plataformas educacionais por sua abordagem descontraída e inovadora, que atende tanto estudantes quanto docentes. Além disso, se destaca por sua postura fora do padrão dos modelos de ensino tradicionais dentro da área computacional.
