%%% TODO: Explicitar a gamificação em cada parte (retomar cap 2)

Neste capítulo, será descrito o aplicativo móvel \appName, assim como todas as suas funcionalidades, telas e aplicabilidade.

O aplicativo móvel foi desenvolvido com o intuito de ensinar computação para qualquer pessoa interessada de forma mais descontraída e gamificada. O projeto foi disponibilizado no GitHub com a licença do MIT \cite{mit-license}, que é uma licença simples e curta com a única condição sendo de preservação do \textit{copyright} e notas da licença. A licença em si é \textit{open-source} e garante a qualquer pessoa, o uso comercial, distribuição, modificação e uso privado do projeto, porém, não provê garantia de forma alguma.

O \appName\ foi dividido em dois módulos, denominados \textit{Back Office} e \textit{Front Office}. O \textit{Front Office} foi o módulo desenvolvido para dispositivos móveis Android e é o elemento principal da aplicação, representado o local onde o aluno irá realizar os desafios, consultar sua colocação, interagir com outros alunos, entre outros. Já o \textit{Back Office} é uma versão Web da aplicação cujo público alvo são professores que queiram utilizar da plataforma para ensinar os seus alunos. Os professores serão capazes de criar cursos, que são compostos por tópicos, que por si, podem possuir diversos desafios, cada um valendo uma quantidade determinável de pontos que os alunos poderão obter. Por se tratar de um aplicativo em que o conteúdo pode ser adicionado a qualquer momento por um professor que se cadastre na plataforma Web, os conhecimentos prévios para o uso não podem ser determinados como um todo, mas sim por curso, a ser determinado pelo professor em questão.

O aplicativo conta com um curso já pré-cadastrado que pode ser acessado pela tela inicial por meio da denominação "Plataforma global". Foram criados exercícios de Python para iniciantes. Para a criação dos exercícios, foi usado como base, o formato dos exercícios de python presentes na plataforma W3schools\footnote{\url{https://www.w3schools.com/}}, porém com diversas adaptações e modificações para adequar à proposta do aplicativo.

As próximas seções mostram as telas e funcionalidades do \appName\, módulo \textit{Front Office} e \textit{Back Office} em mais detalhes.

\section{Front Office}

Versão \textit{mobile} do \appName, possui as seguintes características:

\begin{itemize}
    \item \textbf{Público-alvo}: Estudantes de computação que estejam interessados em aprender de forma gamificada ou alunos de professores que utilizam a plataforma como elemento de aprendizagem de seus respectivos cursos;
    \item \textbf{Plataforma}: Dispositivos móveis, principalmente \textit{smartphones} Android;
    \item \textbf{Objetivo geral}: Aprendizagem gamificada de computação.
\end{itemize}

A seguir, estão listadas as diferentes telas e funcionalidades do \textit{Front Office} do \appName.

\subsection{Login e Registro}

Ao acessar o aplicativo pela primeira vez, será requisitado que o usuário realize o \textit{Login} na plataforma, conforme mostrado na primeira tela da \refFig{login}.

\figura[H]{../img/app/login-pre-home.png}{Telas de \textit{Login}, Registro e Pré-\textit{Home} do aplicativo}{login}{width=\textwidth}

A tela possui os campos de \textit{E-mail} e Senha para que seja possível a validação dos dados para realizar o \textit{login} do usuário ou exibir uma mensagem de erro caso os dados estejam incorretos. Abaixo do botão de \textit{login}, existe um seletor para lembrar do usuário que tem a função de diminuir a incidência de \textit{logins} e também possui um elemento clicável para recuperar a senha de usuário pelo e-mail cadastrado. Por fim, no final da página, encontra-se o botão que leva à tela de registro de usuário. Ao clicar no botão "registre-se aqui", o usuário será levado à tela de registro.

Na tela de Registro, o usuário será capaz de se cadastrar, bastando preencher os dados de nome completo, e-mail, data de nascimento, formação acadêmica e senha, sendo necessária confirmação. O usuário somente poderá se cadastrar se aceitar as políticas de privacidade, que podem ser acessadas pelo \textit{link} presente na \textit{checkbox} no final do formulário. Esta medida serve para deixar o \textit{software} em conformidade com a Lei Geral da Proteção de Dados (LGPD) \cite{lgpd}. O modelo utilizado para a criação da política de privacidade foi disponibilizado gratuitamente na plataforma Jusbrasil pelo usuário Pensador Jurídico \cite{politica-privacidade}.

Com a conta criada e validada corretamente, o aplicativo vai redirecionar para a tela de \textit{login}, onde o usuário deverá digitar as informações de \textit{login} e senha para finalmente acessar as funcionalidades do aplicativo, iniciando pela tela Pré-\textit{Home}. Nesta tela, o estudante poderá escolher a plataforma que irá acessar. Todos os usuários terão acesso à plataforma global, mas os professores podem cadastrar novos times no \appName\ por meio do \textit{Back Office} e matricular estudantes. À medida que o estudante é matriculado nos times, mais opções aparecem na Pré-Home abaixo da plataforma global para o acesso.

\subsection{\textit{Home} e Navegação}
\label{home_navegacao}

Ao acessar o time selecionado, o usuário é redirecionado para a tela \textit{Home} do curso em específico. Nesta tela, existe uma barra de navegação que permite o acesso às telas de \textit{Home}, Conta e Mensagens.

\figura[H]{../img/app/barra_navegacao.png}{Telas acessíveis pela barra de navegação: \textit{Home}, Conta e Mensagens}{navegacao}{width=\textwidth}

Pela tela \textit{Home}, é possível visualizar uma barra de pesquisa para filtrar os tópicos exibidos do curso logo abaixo e o \textit{ranking} de usuários, que é ligado aos pontos obtidos. Pressionando o subtítulo "Tópicos", é possível acessar uma tela que exibe todos os tópicos. O mesmo vale para a seção de \textit{ranking}.

O segundo ícone da barra de navegação leva à tela de Conta, onde o usuário pode visualizar e editar as suas informações pessoais, assim como fazer o \textit{logout}. Ao final da seção em destaque da tela, também existem \textit{links} que levam para \textit{sites} que contém a política de privacidade, perguntas frequentes e ajuda da aplicação.

Já o terceiro ícone, representa a tela de mensagens. Nela, é possível visualizar todas as conversas que foram iniciadas com os diferentes usuários da plataforma. Se houverem mensagens não lidas, um ícone azul com a quantidade de mensagens não lidas irá aparecer ao lado direito do nome do usuário. Selecionando um usuário listado, é possível continuar a conversa com o mesmo por meio da ferramenta de \textit{Chat}.

\subsection{\textit{Ranking} e Pontuação}
\label{ranking}

Conforme mostrado na Seção \ref{home_navegacao}, a tela \textit{Home} exibe o \textit{ranking} do time, porém limitado a dez registros que são ordenados pelos pontos obtidos. Ao pressionar o subtítulo \textit{Ranking}, o usuário é levado à página dedicada ao \textit{ranking} de usuário, onde é possível visualizar todos os participantes do time e suas respectivas posições.

\figura[H]{../img/app/ranking.png}{Telas da funcionalidade de \textit{Ranking}: \textit{Home} e \textit{Ranking}}{ranking}{width=\textwidth}

Os estudantes podem obter pontos por meio dos desafios do curso selecionado, que são acessíveis pelos tópicos (\ref{topics_challenges}). Ao acertar um desafio, o usuário ganha uma quantidade determinada de pontos, que é definida pelo professor que cadastrou o desafio. A quantidade de pontos que podem ser configurados para um desafio é limitada no \textit{Back Office} (\ref{backoffice}).

\subsection{Dados de Usuário}

Os usuários do aplicativo são capazes de alterar os seus dados de usuário por meio da tela de Conta, pelo botão "Editar dados pessoais". Para alterar a senha, existe o botão "Alterar Senha".

\figura[H]{../img/app/conta.png}{Telas para gerenciamento de conta}{conta}{width=\textwidth}

Nas telas de alterar dados e senha, o usuário tem a opção de efetivar a alteração realizada, assim como a opção de descartá-las por meio do botão de voltar no canto superior esquerdo. Todas as opções disponíveis e os dados coletados estão descritos nas políticas de privacidade do aplicativo, assim como é descrita a incapacidade do usuário de deletar a sua conta diretamente ou alterar o endereço de \textit{e-mail}. Porém, nos \textit{links} disponíveis ao final da página, está descrito como proceder para a exclusão da conta e alteração de \textit{e-mail}, que seria por meio do contato por \textit{e-mail} com os responsáveis pelo aplicativo.

\subsection{Conversa}

A funcionalidade de \textit{Chat} da plataforma serve para intermediar a conversa por mensagens entre dois usuários. Para iniciar uma conversa com uma pessoa, basta encontrar e selecionar a pessoa na tela de ranking (\ref{ranking}). Desta forma a conversa é criada e a tela de Conversa com esta pessoa é aberta. A partir do momento em que se iniciou uma conversa com a pessoa, seu nome irá aparecer na tela de conversas para mais fácil acesso do histórico de mensagens.

\figura[H]{../img/app/chat.png}{Telas para a funcionalidade de \textit{Chat}}{chat_screens}{width=\textwidth}

A segunda das telas listadas sobre a funcionalidade de \textit{Chat} mostra uma conversa recém-iniciada, sem nenhuma mensagem. Já a última tela mostrada, inclui uma mensagem de cada uma das partes, demonstando o layout e funcionamento das conversas no aplicativo.

\subsection{Tópicos e Desafios}
\label{topics_challenges}

A funcionalidade de desafios pode ser acessada por meio dos tópicos cadastrados em um time. Cada tópico possui seus próprios conteúdos que podem ser acessados para que o usuário resolva os desafios.

\figura[H]{../img/app/topics.png}{Telas relacionadas à funcionalidade de Tópicos}{topic_screens}{width=\textwidth}

Pela tela Home, ao pressionar o subtítulo "Tópicos", o usuário será redirecionado para uma tela que mostrará todos os tópicos cadastrados do time, permitindo a filtragem por meio da barra de pesquisa ao topo. Ao acessar um tópico pela tela \textit{Home} ou pela tela de Tópicos, o aluno poderá encontrar subtópicos em uma tela dedicada ao tópico selecionado. Cada um dos subtópicos possuem uma quantidade de desafios definida pelo professor, que podem ser acessados por meio do botão presente em cada item e o progresso de resolução é mostrado na barra de progresso acima do botão. Uma vez que o estudante resolve acessar os desafios, ele será direcionado para a tela de desafio, onde deverá responder a primeira questão do subtópico ou passar para o próximo e deixar o problema para depois.

\figura[H]{../img/app/challenge.png}{Telas de Desafio e \textit{Feedback}}{challenge_screen}{width=\textwidth}

O estudante poderá navegar entre os desafios do subtópico para resolver na ordem que achar pertinente e, ao responder ou pular a última questão, ele será direcionado para o \textit{Feedback} deste desafio, onde verá um resumo do seu desempenho. Nos resultados, o estudante será capaz de visualizar a sua taxa de acerto e erro e a quantidade de questões em branco, que ele poderá resolver depois. Também serão mostrados dados de pontuação obtida em cada desafio abaixo do resumo geral, que contém a quantidade total de pontos obtida.

A qualquer momento o usuário é capaz de acessar os desafios já resolvidos e visualizar a resposta que marcou e qual seria a resposta correta, porém, não será possível alterar as respostas uma vez que elas foram submetidas. Assim que o desafio é finalizado, o \textit{ranking} é atualizado com a nova pontuação do usuário, podendo alterar as colocações dos estudantes, promovendo a gamificação (\refCap{2_Gamificacao}).

\section{Back Office}
\label{backoffice}

Versão \textit{Web} do \appName, possui as seguintes características:

\begin{itemize}
    \item \textbf{Público-alvo}: Professores que optarem por utilizar o \appName\ como recurso tecnológico em suas práticas docentes;
    \item \textbf{Plataforma}: Navegadores de Internet;
    \item \textbf{Objetivo geral}: Cadastrar e gerenciar equipes, alunos, desafios e cursos na plataforma \appName .
\end{itemize}

A seguir, estão listadas as diferentes telas e funcionalidades do \textit{Back Office} do \appName .

\subsection{Login e Registro}
Ao acessar a aplicação \textit{Web}, o usuário se depará com a tela de \textit{Login}, componente fundamental para certificar que o usuário é um membro válido da plataforma.

\figura[H]{../img/backoffice/login.png}{Tela de Login - Backoffice}{backoffice_login}{width=\textwidth}

Na tela de \textit{Login}, o usuário informará o seu \textit{E-mail} e Senha para que sejam feitas as validações necessárias dos seus dados. Caso os dados informados sejam válidos, o usuário será direcionado à tela Home da aplicação, conforme a figura X. Se as informações inseridas não condizem com os dados cadastrados na base de dados, será exibido um alerta de erro. 

Abaixo do botão de \textit{Login}, encontra-se o botão que direciona o usuário à tela de registro, conforme a figura \refFig{backoffice_register}.

\figura[H]{../img/backoffice/register.png}{Tela de Cadstro - Backoffice}{backoffice_register}{width=\textwidth}

Na tela de Registro, o usuário encontrará o formulário para cadastro de conta, necessitando preencher os dados: nome completo, \textit{e-mail}, instituição de ensino, senha e confirmação de senha. Caso o usuário deseje prosseguir com o cadastro, deverá aceitar as políticas de privacidade, que podem ser acessadas pelo link presente na \textit{checkbox} da parte inferior do formulário. Esta medida serve para deixar o software em conformidade com a Lei Geral da Proteção de Dados (LGPD) \cite{lgpd}. O modelo utilizado para a criação da política de privacidade foi disponibilizado gratuitamente na plataforma Jusbrasil pelo usuário Pensador Jurídico\footnote{\url{https://pensadorjuridico.jusbrasil.com.br/}} \cite{politica-privacidade}

Após clicar no botão de Cadastro, os dados informados pelo usuário serão validados. Caso os dados apresentados estejam inválidos, será exibido um alerta de erro. Caso contrário, o usuário será redirecionado para a tela de \textit{Login}, podendo acessar a aplicação por meio dos dados recém-cadastrados.

\subsection{Home e Navegação}
