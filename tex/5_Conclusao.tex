%%% TODO: Comentar sobre os objetivos

% \section{Objetivos Específicos}
% A fim de atender os objetivos finais do projeto, foram estabelecidos os seguintes objetivos específicos:
%  \begin{itemize}
%    \item Entender o perfil dos estudantes de computação;
%    \item Compreender os impactos da gamificação no processo de aprendizagem;
%    \item Construir plataforma digital, visando melhorar o ensino e aprendizagem na área de computação;
%  \end{itemize}

O ensino de computação tem a capacidade de abrir muitas portas para o futuro dos profissionais da área e até para o Brasil como um todo. O setor de TIC tem se tornado, cada vez mais, uma das áreas mais importantes do desenvolvimento do país, pois trata da aplicação da tecnologia na sociedade e é uma área que está precisando de cada vez mais profissionais qualificados. Apesar disso, as taxas de evasão dos cursos relacionados à computação se mostram altas, que pode ter como causa, diversos fatores. O \appName\ é um aplicativo móvel que procura aproximar mais pessoas do ensino de computação por meio da gamificação, disponibilizando ferramentas para os professores adequarem as suas aulas ao modelo de forma a aumentar a motivação e o engajamento dos alunos.

% Comentar sobre testes exploratórios (se houver)

Seguindo a metodologia Scrum, ao final de cada \textit{Sprint}, foi desenvolvido um protótipo funcional para avaliação do produto atual. A versão atual representa uma versão funcional da ideia do projeto, que ainda tem muito a evoluir com novas \textit{Sprints} a serem desenvolvidas para realizar melhorias dos mais diversos aspectos para tornar a experiência do usuário mais agradável e potencializar o fator da gamificação. O processo de desenvolvimento foi feito utilizando conceitos de metodologias ágeis, porém foi desenvolvida documentações necessárias e que facilitam a manutenção e compreensão de cada módulo desenvolvido. Os testes automatizados \cite{tdd} são uma ótima fonte para tentar garantir que todas as funcionalidades continuarão funcionando independente de quaisquer ações evolutivas ou de manutenção. Também foram criados e utilizados diversos componentes por meio do React para modularizar o aplicativo, assim como promover o reaproveitamento.

\section{Trabalhos Futuros}

Como trabalhos futuros, pretendemos melhorar principalmente a questão da gamificação e experiência de usuário para que o \appName\ se torne uma ferramenta completa de ensino e aprendizagem de computação. O uso de técnicas de UX \cite{ux} e a inclusão de mais elementos de gamificação \cite{gamification_motivates} como uma loja no aplicativo para gastar os pontos comprando avatares ou outros itens, trazem a possibilidade de aumentar a motivação do estudante.
%Por fim, o teste da plataforma em sala de aula por professores com uma boa experiência na área que estejam abertos às ideias de ensino inclusivo por meio da gamificação ajudariam em obter resultados mais concretos e \textit{feedbacks} de alunos e professores sobre possíveis melhorias mais diretas a implementar.
