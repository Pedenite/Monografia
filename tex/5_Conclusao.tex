% \item Desenvolver um aplicativo que integre elementos de gamificação em um ambiente de ensino de computação para aumentar a motivação e o envolvimento dos alunos;
% \item Fornecer uma ferramenta para que os professores usem a tecnologia de maneira efetiva e integrem-na em sua prática docente para melhorar a aprendizagem dos alunos e tornar o ensino mais eficiente e interessante;
% \item Oferecer um mecanismo de feedback imediato e eficaz para os alunos;
% \item Acompanhar o progresso dos alunos e fornecer relatórios detalhados aos professores;
% \item Desenvolver uma interface intuitiva e atraente para o aplicativo de ensino gamificado;

Tendo em vista os objetivos específicos deste trabalho, sendo o primeiro deles, o desenvolvimento de um aplicativo que integre elementos de gamificação em um ambiente de ensino de computação, foi desenvolvido com êxito o \textit{Front Office} do \appName. Já o segundo objetivo, de fornecer uma ferramenta para que os professores usem a tecnologia de maneira efetiva e integrem-na em sua prática docente, foi cumprido por meio do \textit{Back Office}. Para o terceiro objetivo, de oferecer um mecanismo de feedback imediato e eficaz para os alunos, ao concluir um desafio no \textit{Front Office}, o aluno tem imediatamente acesso à sua resposta, assim como as taxas de acerto e erro e uma justificativa sobre as respostas corretas. No \textit{Back Office}, o professor é capaz de visualizar os relatórios de desempenho dos alunos, sendo capaz de visualizar as taxas de acerto e erro para cada desafio criado, satisfazendo o quarto objetivo, para permitir o acompanhamento do progresso dos alunos e fornecer relatórios detalhados aos professores. Por fim, para o útimo objetivo, de desenvolver uma interface intuitiva e atraente para o aplicativo de ensino gamificado, foram desenvolvidos protótipos das telas usando recursos do Figma, que facilitam o \textit{design}, consequentemente, melhorando a experiência dos usuários.

O ensino de computação tem a capacidade de abrir muitas portas para o futuro dos profissionais da área e até para o Brasil como um todo. O setor de TIC tem se tornado, cada vez mais, uma das áreas mais importantes do desenvolvimento do país, pois trata da aplicação da tecnologia na sociedade e é uma área que está precisando de cada vez mais profissionais qualificados. Apesar disso, as taxas de evasão dos cursos relacionados à computação se mostram altas, que pode ter como causa, diversos fatores. O \appName\ é um aplicativo móvel que procura aproximar mais pessoas do ensino de computação por meio da gamificação, disponibilizando ferramentas para os professores adequarem as suas aulas ao modelo de forma a aumentar a motivação e o engajamento dos alunos.

% Comentar sobre testes exploratórios (se houver)

Seguindo a metodologia Scrum, ao final de cada \textit{Sprint}, foi desenvolvido um protótipo funcional para avaliação do produto atual. A versão atual representa uma versão funcional da ideia do projeto, que ainda tem muito a evoluir com novas \textit{Sprints} a serem desenvolvidas para realizar melhorias dos mais diversos aspectos para tornar a experiência do usuário mais agradável e potencializar o fator da gamificação. O processo de desenvolvimento foi feito utilizando conceitos de metodologias ágeis, porém foi desenvolvida documentações necessárias e que facilitam a manutenção e compreensão de cada módulo desenvolvido. Os testes automatizados \cite{tdd} são uma ótima fonte para tentar garantir que todas as funcionalidades continuarão funcionando independente de quaisquer ações evolutivas ou de manutenção. Também foram criados e utilizados diversos componentes por meio do React para modularizar o aplicativo, assim como promover o reaproveitamento.

\section{Trabalhos Futuros}

Como trabalhos futuros, pretendemos melhorar principalmente a questão da gamificação e experiência de usuário para que o \appName\ se torne uma ferramenta completa de ensino e aprendizagem de computação. O uso de técnicas de UX \cite{ux} e a inclusão de mais elementos de gamificação \cite{gamification_motivates} como uma loja no aplicativo para gastar os pontos comprando avatares ou outros itens, trazem a possibilidade de aumentar a motivação do estudante. No \textit{Back Office}, a inclusão de mais opções para o professor monitorar o desempenho dos alunos e gerar relatórios mais específicos poderia ajudar nas metodologias do professor.
Por fim, o teste da plataforma em sala de aula por professores com uma boa experiência na área que estejam abertos às ideias de ensino inclusivo por meio da gamificação ajudariam em obter resultados mais concretos e \textit{feedbacks} de alunos e professores sobre possíveis melhorias mais diretas a implementar.
