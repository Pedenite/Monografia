O ensino de computação tem a capacidade de abrir muitas portas para o futuro dos profissionais da área e até para o Brasil como um todo. O setor de TIC tem se tornado, cada vez mais, uma das áreas mais importantes do desenvolvimento do país, pois trata da aplicação da tecnologia na sociedade e é uma área que está precisando de cada vez mais profissionais qualificados. Apesar disso, as taxas de evasão dos cursos relacionados à computação se mostram altas, que pode ter como causa, diversos fatores. O \appName\ é um aplicativo móvel que procura aproximar mais pessoas do ensino de computação por meio da gamificação, disponibilizando ferramentas para os professores adequarem as suas aulas ao modelo de forma a aumentar a motivação e o engajamento dos alunos.

% TODO: Comentar sobre testes exploratórios (se houver)

O desenvolvimento da plataforma foi realizado em um curto espaço de tempo de poucos meses e, por isso, o produto entregue não representa um produto final, mas uma versão funcional da ideia, que ainda tem muito a evoluir. O processo de desenvolvimento foi feito utilizando conceitos de metodologias ágeis, porém foi desenvolvida documentações necessárias e que facilitam a manutenção e compreensão de cada módulo desenvolvido. Os testes automatizados são uma ótima fonte para garantir que todas as funcionalidades continuarão funcionando independente de quaisquer ações evolutivas ou de manutenção e também foram criados e utilizados diversos componentes por meio do React para modularizar o aplicativo, assim como promover o reaproveitamento. Tudo isso facilita o incremento da ferramenta para que se chegue em uma versão definitiva.

Na seção a seguir serão discutidas as possibilidades de incremento e evolução do \appName\ com mais detalhes.

\section{Trabalhos Futuros}

Para se chegar a uma versão definitiva do \appName\ para o aprendizado efetivo de computação, seria necessário principalmente, a inclusão de cursos feitos por professores com uma boa experiência na área que estejam abertos às ideias de ensino inclusivo por meio da gamificação. Outros aspectos que poderiam melhorar o aplicativo e a aplicação \textit{Web} seria a inclusão de técnicas de UX e mais elementos de gamificação como uma loja de pontos onde seria possível liberar avatares ou outros itens se utilizando dos pontos obtidos.
